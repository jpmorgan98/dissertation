

\part{Monte Carlo Methods}

\label{part:mc}

\chapter{Introduction to Monte Carlo Methods}

\label{chap:mc_methods_intro}

%Predicting how neutrons move through space and time is important when modeling inertial confinement fusion systems, pulsed neutron sources, and nuclear criticality safety experiments, among other systems.

While deterministic schemes produce an exact solution to an inexact problem, Monte Carlo methods provide an inexact solution---with an associated error---to an exact problem.
Monte Carlo methods can treat the independent variables of the NTE (space, angle, time, energy) as continuous, thus eliminating the discretization errors seen with deterministic methods.
The behavior of neutrons can be modeled with a Monte Carlo simulation, where pseudo-particles with statistical importance are created and transported to produce a particle history \cite{lewis_computational_1984}.
A particle's path and the specific set of events that occur within its history are governed by pseudo-random numbers, known probabilities (e.g., from material data), and known geometries.
Data about how particles move and/or interact with the system are tallied to solve for parameters of interest with an associated statistical error from the Monte Carlo process. 
The analog Monte Carlo method is slow to converge (with a convergence rate of $\mathcal{O}(1/\sqrt{n})$ where $n$ is the number of simulated particles).
New Monte Carlo schemes could converge the solution faster in wall-clock time with fewer simulated particles and may be needed to effectively simulate some systems.

%history with interesting citations
%Monte Carlo methods where orignal propased by Stanislaw Ulam in 1946 after work on the Manhattan project.
%It's orginal use in fact was the tracking neutrons thru phase space using handheld computers called Fermiacs.
%The challenge problem I seek to investate also dates back this time.
%Since then from a report in 1995 over 60\% of computational time on US Government super computing systems was concerend with converging the Monte Carlo method

% deception of the section
My work with Monte Carlo schemes---as with my deterministic work---is broadly divided into two categories
\begin{enumerate}
    \item How to use software engineering libraries to implement work more efficiently (RQ 1); and
    \item Novel methods to converge the solution faster on modern hardware (RQ 5).
\end{enumerate}
The first section digs into the first goal, examining performance portability schemes in high-level languages and the work that is currently deployed in Monte Carlo/Dynamic Code (MC/DC) as part of my work with the Center for Exascale Monte Carlo Neutron Transport (CEMeNT).
The second contains work initially done in a production code at Los Alamos National Laboratory and a proposed extension of the scheme the scheme in an implementation in MC/DC.

\subsection{Portability frameworks for Monte Carlo methods}

In this section I introduce initial investigations into high-level performance portability frameworks.
Developing software to simulate physical problems that demand HPC is difficult.
Modern HPCs commonly use both CPUs and GPUs from various vendors.
Years can be spent porting a code from CPUs to run on GPUs, then again when moving from one GPU vendor to the next \cite{pozulp_progress_2023}.
Portability issues compound when designing software for rapidly developing numerical methods where algorithms need to be both implemented and tested at scale.
Finding a software engineering approach that balances the need for portability, rapid development, open collaboration, and performance can be challenging especially when numerical schemes do not rely on operations commonly implemented in libraries   (i.e., linear algebra as in LAPACK or Intel MKL). 

Common HPC software engineering requirements are often met using a Python-as-glue-based approach, where peripheral functionality (e.g., MPI calls, I/O) is implemented using Python packages but compiled functions are called through Python's C-interface where performance is needed.
Python-as-glue does not necessarily assist in the production of the compiled compute kernels themselves---what the Python is gluing together---but can go a long way in simplifying the overhead of peripheral requirements of HPC software.
With this technique, environment management and packaging uses \texttt{pip}, \texttt{conda}, or \texttt{spack}, input/output with \texttt{h5py}, MPI calls with \texttt{mpi4py}, 
and automated testing with \texttt{pytest}, which can all ease initial development and continued support for these imperative operations. 

Many tools have been developed to extend the Python-as-glue scheme to allow producing single-source compute kernels for both CPUs and GPUs.
% a DSL, pyfr
One tactic is to use a domain-specific language to avoid needing a low-level language (e.g., FORTRAN, C).
A domain-specific language is designed to alleviate development difficulties for a group of subject-area experts and can abstract hardware targets if defined with that goal.
%It can even abstract hardware targets if it is defined with that goal.
PyFR, for example, is an open-source computational fluid dynamics solver that implements a domain-specific language plus Python structure to run on CPUs and Nvidia, Intel, and AMD GPUs~\cite{pyfrPetascale}. 
%The overhead of this Python glue is less than 1\% in PyFR.
Witherden et al.~\cite{pyfrPetascale} discussed how this scheme allows PyFR developers to rapidly deploy numerical methods at deployment HPC scales and have demonstrated performance at the petascale.

Other projects have addressed the need to write user-defined compute kernels entirely in Python script.
Numba is a compiler that lowers a small subset of Python code with NumPy arrays and functions into LLVM, then just in time (JIT) compiles to a specific hardware target \cite{lam_numba_2015}. 
Numba also compiles global and device functions for Nvidia GPUs from compute kernels defined in Python.
API calls are made through Numba on both the Python side (e.g., allocate and move data to and from the GPU) and within compiled device functions (e.g., to execute atomic operations).
When compiling to GPUs, Numba supports an even smaller subset of Python, losing most of the operability with NumPy functions.
If functions are defined using only that smallest subset, Numba can compile the same functions to CPUs or GPUs, or execute those functions purely in Python.
Numba data allocations on the GPU can be consumed and ingested by functions from CuPy if linear-algebra operations are required in conjunction with user-defined compute kernels.
When targeting use of a Python portability scheme to HPC for neutron transport I compared the same transient Monte Carlo neutron transport algorithm in various implementations using PyKokkos \cite{AlAwarETAL21PyKokkos}, PyCUDA/PyOpenCL \cite{kloeckner_pycuda_2012}, and Numba \cite{morgan2022}.
After these initial investigations a Numba+mpi4py software engineering scheme was deemed the most viable for implementation in MC/DC.

%Numba has been shown to be slower then other high level portability frameworks for unoptimized matrix multiplication \cite{Godoy_2023}.
%Monte Carlo neutronic workflows are so memory bound that it's doubtful even significant changes to FLOP performance of a 


%
%I found that all three methods produced similar runtimes for our workflows on CPUs and GPUs for a simple transient Monte Carlo neutron transport simulation \cite{morgan2022}.
%Ultimately, we decided to use a Numba + mpi4py development scheme to build out a Monte Carlo neutron transport code for rapid numerical methods development, portable to various HPC architectures \cite{variansyah_mc23_mcdc,morgan_monte_2024,transport_cement_mcdc_2024} (RQ 1).

\subsection{Delta tracking}

This section describes a variance reduction technique that propose to implement in MC/DC.
Woodcock, or delta, tracking \cite{woodcock_techniques_1965} is a variance-reduction technique that computes the majorant cross-section for the whole problem space, then uses it to determine a distance to collision for all particles.
Coupled with rejection sampling to sort for phantom collisions, and a collision estimator to compute scalar flux, delta tracking often improves performance over analogue Monte Carlo in problems that warrant it (problems with a long mean free path).
Many production Monte Carlo Neutron transport codes like Serpent \cite{leppanen_development_2013, leppanen_use_2017, leppanen_2010_burnup} and others \cite{delta2017rowland} use this method.
In traditional delta tracking first the macroscopic majorant cross section is computed for the entire problem space
\begin{equation}
    \label{eq:majorant}
    \Sigma_{M}(E) = \max\left(\Sigma_{b}(E), ..., \Sigma_{B}(E)\right) \,\text{,}
\end{equation}
where $E$ is energy, $\Sigma_{M}$ is the microscopic majorant cross-section, and $\Sigma_{b}$ is the total microscopic cross-section of the $=b^{\text{th}}$ material.
Now to sample a the distance to a collision
\begin{equation}
    \label{eq:sample}
    D = \frac{-\ln{\xi}}{\Sigma_{M}(E)} \, \text{,} 
\end{equation}
where $\xi\in[0,1]$.
If the potential collision occurs, we move the particle to the sampled distance and do rejection sampling, since we are now potentially forcing collisions that did not occur.
We sort out these phantom collisions by allowing particles to continue to a new sampled distance if
\begin{equation}
    \label{eq:reject}
    \xi < \frac{ \Sigma_{j}(E) } { \Sigma_M(E) } \, \text{,}
\end{equation}
where $\xi\in[0,1]$ is a new random number and $\Sigma_{j}(E)$ is the total macroscopic cross-section of the material ($j^{th}$) where the particle currently resides.
Standard delta tracking is required to use a collision estimator which is less efficient then the normal track length estimator used in surface tracking~\cite{mc2018}.
My goal is to find a way to use the track-length estimator while doing delta tracking which may improve the performance of a Monte Carlo code (RQ 5).


\input{monte_carlo/joss_paper/joss_paper}


%!TEX root = thesis.tex

\newpage
\renewcommand{\TheTitle}{Performance Portable Monte Carlo Neutron Transport in MCDC via Numba}
\renewcommand{\TheAuthors}{Joanna Piper Morgan,
  Ilham Variansyah,
  Braxton Cuneo,
  Todd S. Palmer,
  Kyle E. Niemeyer,}

\renewcommand{\TheAddress}{
\textit{IEEE Computing in Science and Engineering (CISE)} \\
Vol.~VOLUME, PAGES, 2025. \\
\doi{10.1109/MCSE.2025.3550863}
}

\PaperHeader{\TheTitle}{\TheAuthors}{\TheAddress}

\chapter{\TheTitle}
\label{chapter:cise_paper}

\epigraphhead[10]{\singlespacing
    \epigraph{
        The stupidest person on God's green earth is the 2nd lieutenant in the US army. The only person who might rival that would be an engineer right out of school
    }
    {Evan H Morgan (my grandfather)}
}

\section*{Abstract}
Finding a software engineering approach that allows for portability, rapid development, and open collaboration for high-performance computing on GPUs and CPUs is a challenge. 
We implement a portability scheme using the Numba compiler for Python in Monte Carlo / Dynamic Code (MC/DC), a new neutron transport application for rapidly developing Monte Carlo. 
Using this scheme, we have built MC/DC as an application that can run as a pure Python, compiled CPU, or compiled GPU solver. 
In GPU mode, we use Numba paired with an asynchronous GPU scheduler called Harmonize to increase GPU performance. We present performance results (including weak scaling up to 256 nodes) for a time-dependent problem on both CPUs and GPUs and compare favorably to a production C++ code.

\section{Introduction}

Developing software to simulate physical problems that demand high- performance computing (HPC) is difficult.
Modern HPC systems commonly use both CPUs and GPUs from various vendors.
Years can be spent porting a code from CPUs to run on GPUs, then again when moving from one GPU vendor to the next \cite{pozulp_progress_2023}.

Portability issues compound when designing software for rapidly developing numerical methods where algorithms need to be both implemented and tested at scale.
Finding a software engineering approach that balances the need for portability, rapid development, open collaboration, and performance can be challenging especially when numerical schemes do not rely on operations typically implemented in libraries (i.e., linear algebra as in LAPACK or Intel MKL). 

HPC software engineering requirements can be met using a Python-as-glue-based approach, where peripheral functionality (e.g., MPI calls, I/O) is implemented using Python packages but compiled functions are called through Python's C-interface where performance is needed.
Python-as-glue does not necessarily assist in producing the compiled compute kernels themselves---what the Python is gluing together---but can go a long way in simplifying the overhead of peripheral requirements of HPC software.
With this technique, environment management and packaging uses \texttt{pip}, \texttt{conda}, or \texttt{spack}, input/output with \texttt{h5py}, MPI calls with \texttt{mpi4py}, 
and automated testing with \texttt{pytest}, which can all ease initial development and continued support for these imperative operations. 

Many tools have been developed to extend the Python-as-glue scheme to allow producing mostly single-source compute kernels for both CPUs and GPUs.
One tactic is using a domain-specific language to avoid needing a low-level language (e.g., FORTRAN, C).
A domain-specific language is designed to alleviate development difficulties for a group of subject-area experts and can abstract hardware targets if defined with that goal.
PyFR, for example, is an open-source computational fluid dynamics solver that implements a domain-specific language plus Python structure to run on CPUs and Nvidia, Intel, and AMD GPUs~\cite{pyfrPetascale}. 
Witherden et al.~\cite{pyfrPetascale} discussed how this scheme allows PyFR developers to rapidly deploy numerical methods at deployment HPC scales and have demonstrated performance at the petascale.

Other projects have addressed the need to write user-defined compute kernels entirely in Python script.
Numba is a compiler that lowers a small subset of Python code with NumPy arrays and functions into LLVM, then just in time (JIT) compiles to a specific hardware target \cite{lam_numba_2015}. 
Numba can also compile global and device functions for Nvidia GPUs from compute kernels defined in Python.
API calls are made through Numba on both the Python side (e.g., allocate and move data to and from the GPU) and within compiled device functions (e.g., to execute atomic operations).

When compiling to GPUs, Numba supports an even smaller subset of Python, losing most of the operability with NumPy functions.
If functions are defined using only that smallest subset, Numba can compile the same functions to CPUs or GPUs, or execute those functions purely in Python.
Numba data allocations on the GPU can be consumed and ingested by functions from CuPy if linear-algebra operations are required in conjunction with user-defined compute kernels.

Motivated by Numba's capabilities, we developed a new Monte Carlo neutron transport application called Monte Carlo / Dynamic Code\footnote{\url{https://github.com/CEMeNT-PSAAP/MCDC}} (MC/DC) \cite{morgan_monte_2024, variansyah_mc23_mcdc}.
Our development of MC/DC uses a Numba+Python development scheme along with a GPU event scheduler to abate portability issues and allow for rapidly developing novel numerical methods at the HPC scale on CPUs and GPUs.

In this paper, we first introduce neutron transport and the Monte Carlo solution method.
We next describe in greater detail MC/DC's novel (for the field) software engineering approach on CPUs and GPUs, along with pitfalls and difficulties when using this development scheme.
We discuss how novel numerical methods are prototyped and developed in MC/DC, and supported for execution on both GPUs and CPUs.
Then, we analyze the compute performance of MC/DC and, where possible, compare it against modern production Monte Carlo neutron transport solvers.
Finally, we provide concluding remarks and outline future work.

\section{Monte Carlo Neutron Transport and MC/DC}

Predicting how neutrons move through space and time is important when modeling inertial confinement fusion systems, pulsed neutron sources, and nuclear criticality safety experiments, among other systems.
Unlike charged particles, neutrons can interact with the nucleus of an atom because they are unaffected by the negatively charged orbital electrons and the positively charged core.
Some isotopes readily absorb neutrons into the nucleus, which may make such atoms unstable.
When an unstable atom fissions, it releases energy along with two daughter nuclei and subatomic particles, which may be more neutrons, depending on the parent atom.
If additional neutrons are released and encounter more fissionable material, the release of subsequent neutrons can induce a chain reaction.
Thus, a population of neutrons can change rapidly in dynamic systems.

Simulating neutron transport problems is computationally difficult using any numerical method, because the neutron distribution is a function of seven independent variables: three in space, three in velocity, and time \cite{lewis_computational_1984}.
Modern HPC systems now enable high-fidelity simulation of neutron transport for problem types that have seldom been modeled before due to limitations of previous computers. % need citation?
Specifically, large-scale, highly dynamic transport problems require thousands of compute nodes using modern hardware accelerators (i.e., GPUs) \cite{hamilton_continuous-energy_2019, romano_openmc_2015}.

The behavior of neutrons can be modeled with a Monte Carlo simulation, where particles with statistical importance are created and transported to produce a particle history \cite{lewis_computational_1984}.
A particle's path and the specific set of events that occur within its history are governed by pseudorandom numbers, known probabilities (e.g., from material data), and known geometries.
Data about how particles move and/or interact with the system are tallied to solve for parameters of interest with an associated statistical error from the Monte Carlo process.

The analog Monte Carlo method converges slowly at a rate of $\mathcal{O}(1/\sqrt{n})$, where $n$ is the number of simulated particles.
New Monte Carlo schemes could converge the solution faster in wall-clock time with fewer simulated particles and may be needed to effectively simulate some systems.
We wrote MC/DC to enable rapidly developing these novel numerical methods for time-dependent simulations in particular.


\begin{figure*}
    \centerline{\includegraphics[width=.95\textwidth]{monte_carlo/cise_paper/cise_figs/flow.pdf}}
    \caption{MC/DC's overall structure and how functions get called and interact. Green functions are entirely Python, and black functions are compiled compute kernels if they are running in Numba CPU or GPU modes.}\vspace*{-5pt}
    \label{mpi_mcdc}
\end{figure*}

MC/DC offers a similar feature set as other Monte Carlo neutron transport applications (e.g., OpenMC \cite{romano_openmc_2015}, Shift \cite{hamilton_continuous-energy_2019}) with support for $k$-eigenvalue and fully time-dependent simulation modes in full three-dimensional constructive solid geometry.
It can model the neutron distribution in energy using either continuous energy or multigroup nuclear data.
It also supports domain decomposition.
All these features are supported on CPU (x86, ARM, POWERPC) and GPU processor targets (Nvidia and AMD), with MPI to target multiple processors (via mpi4py \cite{dalcin_mpi4py_2021}).

The number of novel schemes and simulation techniques implemented in MC/DC in a short time illustrates the success in its software engineering structure.
MC/DC supports the use of the iterative quasi-Monte Carlo (iQMC) method where deterministic and Monte Carlo transport operations run in tandem to converge solutions faster than they would in a pure Monte Carlo method. %\cite{pasmann_iqmc_nodate}
Other novel developments include global sensitivity analysis, hash-based random number generation for fully replicable solution testing, time-dependent population control, and continuously moving surfaces.
Several ongoing developments include quasi Monte Carlo, residual Monte Carlo, and machine learning techniques for dynamic node scheduling.

\section{MC/DC on CPUs}

To compile on CPUs, MC/DC uses the Numba compiler for Python to lower compute functions into LLVM and compile for a specific hardware target.
Figure~\ref{mpi_mcdc} shows MC/DC's functional layout when running in both MPI and Numba mode.

First, a user writes a Python script and imports \texttt{mcdc} as a package and forms an input script.
Then, the user interfaces with functions described in the input handler within MC/DC describing the physical models, material data, and simulation parameters.
This input layout is similar to how other Monte Carlo neutron transport applications input problems.

The input script calls a run command, which starts initialization functions within MC/DC. 
The initialization process allocates and constructs a global variable containing the user-defined inputs, meshes, particle banks, event tallies, and current global states.
This global variable is formed from a statically typed NumPy \texttt{ndarray}, which acts like a Python dictionary, where keywords are used to extract numerical arrays.
After building the global variable, initialization functions dispatch MPI processes if running in MPI mode, and begin the Monte Carlo neutron transport simulation.


Each MPI process calls functions containing the various transport algorithms and modes that MC/DC supports.
Each transport function is decorated with a Numba JIT (\texttt{@jit}) compilation flag declaring that each function must be compiled before being executed if running in Numba mode.
These transport logic loops are the highest level at which Python functions will be compiled in MC/DC.
For example, a fixed source problem will loop over all the particles and transport them until the particle's history is terminated from a physical event (e.g., capture, fission, time or space boundary), a simulation event (e.g., time census), or a variance-reduction event (e.g., population control, implicit capture).

The specific functions within each algorithm that conduct the actual transport operations (e.g., moving particles, tallying events, generating daughter particles from fission) are contained in the kernel set where all functions are \texttt{@jit} decorated.
Figure~\ref{fig:jitfunctions} shows an example compute kernel that updates the position and time of a particle as it moves.
It also shows the declaration of an \texttt{numpy.ndarray} data structure used in MC/DC.

\begin{figure}
\begin{lstlisting}[language=Python]
import numpy
from numba import jit

part = numpy.dtype([
    ('x', float64), ('y', float64),
    ('z', float64), ('ux', float64),
    ('uy', float64), ('uz', float64),
    ('v', float64) ])

@jit
def move_particle(P: part, distance):
    P['x'] += P['ux'] * distance
    P['y'] += P['uy'] * distance
    P['z'] += P['uz'] * distance
    P['t'] += distance / P['v']
\end{lstlisting}
\caption{Example of a decorated function and MC/DC's data structures based on \texttt{numpy.ndarray}.}
\label{fig:jitfunctions}
\end{figure}

After all transport is completed and the simulation is finished, the program returns to the Python interpreter and calls finalization functions.
Here, requested tally information along with statistical error provided from the Monte Carlo process are saved in an HDF5 file.
Data can be extracted from this HDF5 file and used in Python scripts to do post-processing analysis and/or data visualization with tools like Matplotlib, or post-processing can be done in other applications like Visit or Paraview.

When initially exploring a novel transport method, a developer can work in a pure Python environment where functions are entirely executed in the Python interpreter.
In this mode, the developer can bring any package into any function, do typing dynamically, and use any Python data structure.
MC/DC can be executed in MPI mode in Python as well as compiled CPU mode.
While a full Python development environment is great for initially proving a concept, it often proves to be too slow for problems of interest.

When more performance is required, developers rewrite their kernels to strictly use Numba-enabled functions.
Numba only supports a small subset of the Python ecosystem. 
Some Python data structures like dictionaries and lists can no longer be used and must instead come from NumPy implementations. 
Thus, when using Numba, the small subset of functions supported effectively becomes a domain-specific language.

Scientific computing using Python is often done with NumPy functions and data structures, making these fairly natural for numerical-methods developers to use and understand.
In fact, we have found NumPy functionality to be more commonly used in initial development than other non-supported Python methods, making the restrictions in Numba more palatable.
Some developers report skipping Python-mode development entirely and starting with Numba-CPU work for their initial proofs of concept, as they find that aids in future debugging efforts.
Similarly, other developers report making small, incremental changes in Python-based algorithms, then checking to ensure successful compilation in Numba before moving forward, roughly at every commit.
When kernels are written to support Numba mode, they can be compiled to any supported CPU targets automatically (i.e., x86, ARM64, PPC64).

We can identify pitfalls with this approach, the most significant of which are:
\begin{itemize}
    \item Common failures of \texttt{numba.object\_mode},
    \item Lack of MPI calls from within the JIT-compiled Numba code,
    \item Numba kernel debugging and profiling;
    \item Loss of functions from SciPy not implemented in NumPy, and
    \item Restrictions with \texttt{numpy.ndarray} as our primary data structure.
\end{itemize}
Most of these issues have workarounds, but make implementing numerical methods in Numba harder.

Consider that \texttt{numpy.ndarray} requires ``square'' size allocation for all elements such that the size of every named element within an array must be the same.
If one element requires \num{10000} data points and the next only \num{100}, the size of that \texttt{numpy.ndarray} is \num{20000}, which is a drastic over-allocation.
This is a primary issue for continuous energy material data, where some materials may require tens of thousands of points to fully resolve, and others may only need hundreds.
While Numba does have some features to help in this circumstance (namely experimental \texttt{jit\_classes}), we must keep the \texttt{numpy.ndarray} to support MPI calls and GPU portability.
To fix this issue, given our constraints, we are moving towards using one-dimensional vectors with length information to offset between different variables, potentially impacting MC/DC's developer-friendliness.
Accepting increased complexity to achieve portability is common in MC/DC, so developing in it can be about as difficult as in a low-level language.

Other deficiencies are known to the Numba community, and some even have ongoing open-source remedies.
For example, \texttt{numba-mpi}\footnote{\url{https://github.com/numba-mpi/numba-mpi/}} is a project to support compiled-side MPI calls, \texttt{Profila}\footnote{\url{https://github.com/pythonspeed/profila}} attempts to bring the GNU debugger to Numba kernels, and \texttt{numba-scipy}\footnote{\url{https://github.com/numba/numba-scipy}} extends support for more SciPy functions to Numba.
However, most of these community projects are still in their infancy and not robust enough to handle the large and complex structures in MC/DC.

For CPU-based HPC deployments, a Python-as-glue strategy with Numba compute kernels can enable portable (between CPU architectures and scales) and high-performance code.
However, on GPUs, if using Python+Numba alone, a developer must still have in-depth understanding of their target GPU parallelism paradigm to achieve high performance.

\section{MC/DC on GPUs}

GPUs use a single-instruction multiple-thread (SIMT) parallelism paradigm, where threads are executed in teams called warps, or wavefronts, and do the same operations in lockstep. 
If threads in the same warp need to take different paths in a program (e.g., different if/else branches or iterating loops a different number of times), each path must be executed serially.
This behavior is called thread divergence.
Threads that do not belong to the currently executing path are disabled so that the end result of the computation is consistent with the control flow logic.
Mitigating thread divergence will usually result in higher performance of GPU-enabled applications.

Unfortunately, commonly implemented Monte Carlo neutron-transport algorithms are examples of highly divergent workflows, as the behavior of any individual particle is governed by random numbers.
Much more work is often required beyond naive syntax porting to implement Monte Carlo radiation transport applications to GPUs \cite{pozulp_progress_2023}.
When compiling and running on GPUs, MC/DC uses an open-source asynchronous event scheduling library called Harmonize\footnote{\url{https://github.com/CEMeNT-PSAAP/harmonize}}~\cite{brax2023} to 
reorganize the execution of business logic and storage/movement of data to better fit the SIMT execution paradigm of GPUs.
Harmonize implements runtimes that examine operations due to be executed, segregating them into like-operations so that like-work may be executed together in batches.

Monte Carlo transport functions lend themselves to asynchronous programming schemes, as it is intuitive to provide a function for each particle operation.
For example, Figure~\ref{fig:jitfunctions} shows a \texttt{move\_particle} function.
These functions can be ordered such that like operations get implemented in unison during runtime even if user defined control logic would dictate otherwise.
The end result of the computation is the same, but the order of execution on the processor has been optimized.
MC/DC calls Harmonize via Python bindings.
Harmonize has been shown to increase GPU performance by reducing thread divergence \cite{brax2023}.


\begin{figure}
\begin{minted}[mathescape, linenos]{python}
from numba import cuda

@for_cpu
def add(array, value, idx):
    array[idx] += value

@for_gpu
def add(array, value, idx):
    cuda.atomic_add(array, value, idx)

def tally_collision_event(mcdc, part):
    id = loc2index(part)
    add(mcdc.col_tally, part.v, id)
\end{minted}
\caption{Example of GPU and CPU specific API calls as defined in MC/DC and their use in a collision tally function.}
\label{fig:forcpuvgpu}
\end{figure}

% description of GPU mode execution
Moving to compile and run Numba \texttt{jit}ed functions to the GPU requires making a few alterations to the kernels themselves.
An even-smaller subset of Python functions work in GPU-compiled code, with operations supported on Numba-CPU like \texttt{numpy.linalg.solve()} losing support.
Other operations may require API-specific calls, exposed by Numba commands.
For example, atomic operations are required to preserve the side-effects of individual threads acting on global memory (e.g., adding to a tally).
To allow for a mostly unified kernel base in MC/DC for both CPUs and GPUs, we track alternate function implementations registered through decorators.

Figure~\ref{fig:forcpuvgpu} shows how we implement alternate tally accumulation functions using \texttt{@for\_cpu} and \texttt{@for\_gpu} decorators.
Here \texttt{@for\_cpu} adds one to a value in an array, and since this is within a single MPI rank we can assume a thread-safe operation.
However, on the GPU this may result in a memory race condition requiring an \texttt{numba.cuda.atomic\_add} API call.
While this does increase complexity for a programmer implementing numerical methods, it is nowhere near the complexity that might be required to accomplish a similar implementation in a compiled language.

Most numerical methods development in MC/DC is done by editing pre-existing control flow (e.g., adding more operations or device functions to existing loops, adding more components to a data structure).
Once all alterations can compile and execute using Numba-CPU functionality and necessary API calls have been abstracted, MC/DC and Harmonize automatically compile and execute those extra commands on GPUs.
So, in most cases, methods developers do not need to interface with Harmonize commands or make any alterations to the GPU runtime, data management, or compilation techniques.

If more-significant alterations are required for a given numerical method, a developer may have to interface directly with Harmonize.
We have found that, for the majority of our work exploring new algorithms to date, Harmonize+Numba sufficiently abstracts the SIMT parallelism paradigm such that operations that work on the CPU side are generally supported on GPU with little effort from the methods developer.

% descirption of compilation + harmonize
To compile functions to GPU targets with Harmonize, Numba generates intermediate compiler representations (IRs, e.g., LLVM-IR or PTX) of Monte Carlo neutron transport kernels. 
Harmonize then ingests and links those IRs with the event-scheduling runtime.
MC/DC's documentation\footnote{\url{https://mcdc.readthedocs.io/en/dev/theory/gpu.html}} provides a more in-depth description how MC/DC and Harmonize are JIT compiled for given hardware.

When running MC/DC in GPU mode on an individual MPI thread, MC/DC+Harmonize is first JIT compiled, then  during initialization allocates device memory for the global array and moves this from the host (CPU) to the device (GPU).
Next, MC/DC's transport kernels are executed with Harmonize on the GPU until transport for a given collection of work is complete.
Communication between the GPU and CPU of the global variable may be required during transport for some simulation modes.
When transport is finished, the global variable moves back to the host for a final time, and the simulation completes.
% workflow description

Just as with CPU development, this abstraction strategy has some potential disadvantages.
While MC/DC's software engineering structure allows for kernel portability between CPUs and GPUs significant time and effort can be lost in debugging, particularly for the data structures.
MC/DC only operates on GPUs using Harmonize.
Beyond its event scheduling and runtime capabilities, Harmonize allows us to ameliorate issues in Numba's GPU feature set.
For example, allocating and moving data from the CPU to GPU can only happen from Python code and cannot be done from Numba-compiled CPU kernels (requiring an \texttt{object\_mode} call).
In our initial implementations this required many copies of the global variable, which proved prohibitively costly for larger problems.
Using API calls elevated through Harmonize instead of Numba fixes this issue, requiring only two copies of the data, and the data can be accessed from both Numba-compiled CPU and GPU kernels.
In addition, when extending GPU operability to other vendors (namely AMD GPU support), Harmonize allows us to elevate non-implemented Numba API calls to the MC/DC Python interface.
For example, the Numba-HIP package\footnote{\url{https://github.com/ROCm/numba-hip}} does not currently support atomic operations on vectors.
Harmonize provides a clear path to elevate HIP-C\texttt{++} functions into Python for use in MC/DC.

For GPU development, the portability and performance enabled by MC/DC's software engineering structure increases the difficultly of implementation for the workflow developer who actually interfaces with Numba and Harmonize.
Our hope is that the investment made by the workflow developers is compounded with rapid development of more numerical methods.


\section{Performance}

\begin{figure}
    \centerline{
    \includegraphics[width=.4 \textwidth]{monte_carlo/cise_paper/cise_figs/kobayashi_problem.png}
    } 
    \caption{Kobayashi problem schematic.}
    \label{koby-problem-def}
\end{figure}

\begin{figure*}[h]
    \centerline{
    %\includegraphics[width=.3\textwidth]{monte_carlo/cise_paper/cise_figs/kobayashi_problem.png} \hspace{1cm}
    \includegraphics[width=\textwidth]{monte_carlo/cise_paper/cise_figs/koby.pdf}
    } 
    \caption{Time and space averaged scalar flux solution to the Kobayashi problem run with $1\times 10^{9}$ particle histories at various points in time.}
    \label{koby-results}
\end{figure*}

% transient runtime strong scaling mcdc v openmc maybe v shift?
To examine the performance of MC/DC we use a time-dependent version of the one-group Kobayashi dog-leg void-duct problem \cite{Kobayashi2001, variansyah_mc23_mcdc}.
Figure~\ref{koby-problem-def} shows the void duct and the location of the neutron source at the opening of the duct.
The initial condition is zero flux everywhere.
Radiation quickly moves through the void and then penetrates the walls of the problem, slowly dissipating through time.
Figure~\ref{koby-results} shows the duct clearly with the scalar flux solution at various points in time.

We solved the Kobyashi problem on HPC systems available at Lawrence Livermore National Laboratory (LLNL): the Dane and Lassen machines.
Dane is a CPU-only system with dual-socket Intel Xeon Sapphire Rapids CPUs, each with 56 cores for a total of 112 per node.
Lassen has four Nvidia Tesla V100s and two IBM Power 9 CPUs per node.
To contrast MC/DC on the CPU against a traditionally developed and compiled code, we will compare performance to another Monte Carlo neutron transport code, OpenMC\footnote{\url{https://github.com/openmc-dev/openmc}} \cite{romano_openmc_2015} (an open-source code written in C\texttt{++}).
We added time-dependent functionality to OpenMC\footnote{\url{https://github.com/CEMeNT-PSAAP/openmc/tree/transient}} so that the same algorithm is implemented in both codes for the Kobyashi problem.


Figure~\ref{performance_results} at left shows the wall-clock runtime of OpenMC (112 MPI threads), MC/DC-CPU (112 MPI threads), and MC/DC-GPU (four MPI threads) using all available resources of a given node type.
Both MC/DC runs are JIT compiled, which means compiling consumes a considerable amount of wall-clock runtime for even small problems (about \SI{70}{\s} and \SI{140}{\s} for CPU and GPU targets, respectively).
For small particle counts, actual compute time is small relative to compile time, so both MC/DC lines are flat until enough work saturates the computational power of a given resource---around \num{e8} particles for MC/DC-CPU and \num{e9} for MC/DC-GPU.
At full saturation (\num{e10} particles) MC/DC-CPU runs about 22\% slower than OpenMC, while MC/DC-GPU is 8$\times$ faster than MC/DC-CPU and 6$\times$ faster than OpenMC.

OpenMC displays superior performance at smaller particle counts due to it being a fully compiled code.
GPU profiling for MC/DC shows that
\texttt{memalloc} and \texttt{memcopy} CUDA API calls 
occupies 2.2\% of runtime (\SI{32.8}{\s} out of \SI{1500}{\s})
when running \num{e10} particles on one Lassen GPU for the Kobyashi problem.
At \num{e9} particles, GPU memory commands account for 11.8\% of runtime (\SI{18.0}{\s} out of \SI{147.0}{\s}).


Figure~\ref{performance_results} at right shows weak-scaling performance (\num{e10} particles per node) with the Kobyashi problem for MC/DC-CPU (Dane), OpenMC (Dane), and MC/DC-GPU (Lassen).
Each node is using all available compute resources for a given calculation (e.g., four nodes is 480 CPU cores and 16 GPUs on Dane and Lassen, respectively).
MC/DC-CPU shows a minimum efficiency of \num{0.85} at 256 nodes while OpenMC only falls to \num{0.89}.
OpenMC supports shared-memory parallelism (using OpenMP) but these calculations only use domain-replicated MPI.
MC/DC-GPU shows the best weak-scaling efficiency for this problem, decreasing only to \num{0.95} at 256 nodes.

\begin{figure*}[h]
    \centerline{
    \includegraphics[width=1.1\textwidth]{monte_carlo/cise_paper/cise_figs/comps.pdf}
    }
    \caption{Left: Wall-clock runtime of the Kobyashi problem over particle counts. 
    Right: Weak scaling efficiency as a function of node count for the Kobyashi problem on Dane (CPU) and Lassen (GPU).}
    \label{performance_results}
\end{figure*}

\section{Discussion, Conclusions, and Future Work}

Monte Carlo/Dynamic Code (MC/DC) is a Monte Carlo neutron transport code that targets modern HPC architectures with CPUs and GPUs.
Our performance results demonstrate that MC/DC's structure using a Python + Numba + MPI + Harmonize scheme can produce similar performance to other Monte Carlo neutron transport solvers.
After JIT compilation overhead, MC/DC performs similarly to traditionally compiled production code on a single node for a transient problem of interest.
MC/DC exhibits similar weak scaling on CPUs and superior weak scaling on GPUs up to 256 nodes of a given HPC, compared with a CPU-only production code.

Developing using Numba for CPU targets can be as difficult as developing in low-level languages for the complicated algorithms we implement. 
This agrees with previously published analysis \cite{KailasaSrinath2022PAEi}.
We found developing the necessary time-dependent features to model the Kobyashi problem in OpenMC to be about as difficult as making changes within MC/DC.
For our application, anything gained when using a high-level-language is lost in time and effort spent circumventing unsupported operations and debugging. 
However, the implementation in OpenMC remains CPU-only, while for MC/DC it took little effort to go from a working CPU implementation to something operating and highly-performing on GPUs.
Of course, we use our own specialized event-scheduling library to do this---but Numba allows us to construct a Python-based portability framework fit to our numerical method with the added benefit of unifying our high-level glue language and kernel-production language.

Over the duration of developing MC/DC (starting in 2021) we have seen many improvements to Numba.
Compiler error reporting continues to improve (especially in Numba versions 0.59.0$+$), the number of supported operations have grown, and Numba has been extended to additional accelerators like AMD and Intel\footnote{\url{https://github.com/IntelPython/numba-dpex}} GPUs.
We have found that the Numba development team fosters a supportive community that is approachable and responsive to questions, comments, and concerns.
We believe that as Numba matures we will continue to see performance and development improvements.

Work in MC/DC is ongoing.
We are continually exploring novel variance reduction and hybrid Monte Carlo techniques, and adding new functionality.
For GPU development specifically we are currently investigating use of unified memory between the CPU and GPU as well as extending support to Intel GPUs.
We will continue to improve MC/DC, making it a portable application for rapid methods development enabled by Python and Numba.

\section{Acknowledgments}
The authors thank the Numba development team for support using the Numba compiler as well as Damon McDougall and Dominic Etienne Charrier from Advanced Micro Devices for support using Numba-HIP and ROCm compilers.
The authors thank the high performance computing staff at Lawrence Livermore National Laboratory for continued support using the Dane and Lassen machines.

This work was supported by the Center for Exascale Monte-Carlo Neutron Transport (CEMeNT) a PSAAP-III project funded by the Department of Energy, grant number: DE-NA003967.

\newpage
\renewcommand{\TheTitle}{Hybrid Woodcock-delta Tracking Schemes Using a Track-Length Estimator}
\renewcommand{\TheAuthors}{Joanna Piper Morgan,
  Ilham Variansyah,
  Kayla B. Clements,
  Todd S. Palmer,
  Kyle E. Niemeyer,}

\renewcommand{\TheAddress}{
\textit{Submitted to Journal of Computational and Theoretical Transport} \\

}

\PaperHeader{\TheTitle}{\TheAuthors}{\TheAddress}

\chapter{\TheTitle}
\label{chap:delta_tracking_paper}

\epigraphhead[10]{\singlespacing
    \epigraph{
        And it's knowing I'm not shackled\\
        By forgotten words and bonds\\
        And the ink stains that are dried upon some line\\
        That keeps you in the back roads by the rivers of my memory\\
        That keeps you ever gentle on my mind\\
    }
    {Glenn Campbell}
}

\newcommand{\maj}{\Sigma_{\text{maj}}}

%%% Write text for abstract
%%% Most text modifying commands will work in abstract

\section*{Abstract}

Woodcock-Delta tracking is a common alternative to the more popular surface tracking technique where the largest cross-section at a given energy in the whole problem is used to sample a distance to collision at any point.
This process forces extra nonphysical collisions so it is paired with a rejection sample to determine real events from virtual or phantom events.
Traditional implementations of Woodcock-delta tracking preclude the use of a track (or path-) length estimator for scalar flux tallies and are forced to use the normally higher-variant collision estimator instead.
There is no mathematical reason why the track-length estimator cannot be used in conjunction with Woodcock-delta tracking only implementation issues. 
In this work we take advantage of that to produce a Woodcock-delta tracking algorithm which tallies fluxes to a structured rectilinear mesh using the track-length estimator.
This development more readily enables hybrid surface-delta tracking algorithms as the track-length tally can be used everywhere for scalar flux estimation regardless of which tracking algorithm a particle is using.
We use this when developing a novel hybrid-in-energy method where Woodcock-delta tracking is used in high energies (where mean free paths are long) and surface tracking below that (starting at the neutron resonances) as well as a previously defined hybrid-in-material method.
We verify that delta tracking algorithms we consider can be used in conjunction with continuously moving surfaces.
We benchmark these methods showing figures of merit on four time-dependent problems: two multi-group and two continuous-energy.
Woodcock-delta tracking with a track-length tally showed modest improvements to figures of merit as compared to traditional delta tracking with a collision estimator and surface tracking with a track-length estimator (\num{1.5}$\times$--\num{2.5}$\times$) and significant improvements (\num{7}$\times$--\num{11}$\times$) when using the hybrid-in-energy method.

\section{Introduction}

% general introduction
Predicting the neutron distribution in space, energy and time is important when modeling inertial confinement fusion systems, pulsed neutron sources, and nuclear criticality safety experiments, among other systems.
The behavior of neutrons can be modeled with a Monte Carlo simulation, where particles with statistical importance are created and transported to produce a particle history \cite{lewis_computational_1984}. 
The path of a particle and the specific set of events that occur within its history are governed by pseudorandom numbers, known probabilities (for example, from material data), and known geometries. Data about how
particles move and/or interact with the system are tallied to compute parameters of interest with an associated statistical error from the Monte Carlo process.

There are two common methods used to sample the random walk in a Monte Carlo neutron transport algorithm: surface tracking \cite{lewis_computational_1984} and Woodcock-delta tracking \cite{woodcock_techniques_1965}.
These two tracking algorithms have complementary performance bottlenecks, for a certain class of problems, a hybrid method may allow for greater performance than either approach used individually.
For example: traditional implementations of Woodcock-delta tracking preclude evaluating quantities of interest with a track-length estimator instead opting for a collision estimator may be used, whereas, surface tracking has no such restrictions.
The track-length estimator usually provides a better estimate of quantities of interest (see section \ref{sec:tracking_algs_and_est}).
On the hand, surface tracking requires potentially complicated geometric operations whereas Woodcock-Delta tracking does not.
Which method to use is often problem dependent.

% cite previous M&C paper

Previous work into delta-surface tracking on a structured mesh has shown good performance for problems with complex arrangements of optically-thin materials\cite{morgan2023delta}.
Making material-based decisions about when to do delta and surface tracking has also been explored and implemented in production Monte Carlo codes \cite{leppanen_development_2013conf, leppanen_2010_burnup, richards_monk_2015}.
We extend the idea of \textit{tracking} on a structured mesh to full Woodcock-delta tracking (eliminating the distance to surface check) and \textit{tallying} to a structured mesh allowing the use of a track-length estimator for estimations of scalar flux.
% MC/DC introduction

In this work we describe, verify, and evaluate the performance of a delta tracking algorithm that allows the use of a track-length estimator on a structured tally mesh.
We then use this approach in two hybrid delta tracking schemes: one in which the choice tracking algorithm is based on material region, and another scheme based on the particle energy.
We implement this work in Monte Carlo Dynamic Code (MC/DC), an open source Monte Carlo neutron transport application purpose-built to conduct rapid numerical methods development, specifically for time-dependent problems \cite{morgan_monte_2024}.
We compute figures of merit for four computationally difficult benchmark problems including a stuck rod accident simulation of a continuous energy version of the C5G7 geometry, and runtime results for a whole CPU node (2X Intel x86 Xeon Sapphire Rapids) and a whole GPU Node (4X Nvidia Tesla V100).

This work is novel as it is the first published usage of a track-length estimator for scalar flux estimation with full Woodcock-delta tracking, the first usage of Woodcock-delta tracking in conjunction with continuously moving surfaces (implemented in MC/DC \cite{variansyah_2023_highfidelity}), and the first time a hybrid delta tracking method has been based on the energy of a given particle.
The quantities of interest in this work are estimates of scalar flux and not full reaction rate densities for all particle material interactions.
While this does limit the generally applicability of these methods, scalar flux is a necessary quantity for hybrid methods, and other uses.
Indeed for deterministic method if scalar flux is ascertained the problem is considered solved.


\section{Tracking Algorithms and Estimators}
\label{sec:tracking_algs_and_est}

% Surface tracking
To track the movement of neutrons within a system and, one of two tracking (or sampling) methods are employed. 
The first is surface tracking described in algorithm \ref{alg:surface}. 
For a particle in material $m$, a distance to collision is sample from a cumulative probability distribution function by
\begin{equation}
    d_{\text{collision}} = \frac{-\ln(\xi)}{\Sigma_{t,m}} \; ,
\end{equation}
where $\Sigma_{t,m}$ [\SI{}{\per\centi\meter}] is the macroscopic total cross section of the $m$-th material and $\xi$ is a pseudo-random number between zero and one.

This sampling of the cumulative probability distribution function will only hold true while the material is homogeneous.
So if the distance to collision is beyond a material interface in a system with multiple materials, the particle must be stopped at that interface surface and a new distance to collision must be calculated with with the new material's $\Sigma_{t,m}$.
This approach is an unbiased way of dealing with the sampling of distance to collision in a heterogeneous medium.
In a standard surface tracking algorithm this involves computing both a distance to collision ($d_{\text{collision}}$) and a distance to the nearest surface along the particle's direction of travel ($d_{\text{surface}}$).
The smaller of these two distances determines which event happens to the particle - a collision or a surface-crossing.
After or while the particle is moving, tallies can be accumulated to compute quantities of interest.
If a collision occurs, more sampling and operations can be done (e.g., isotropically scatter a particle).
If the particle is still alive at the end, the algorithm is repeated.
The distance to nearest surface computation can become quite expensive as geometries grow in complexity (e.g. complex CSG geometries or CAD based surfaces).
Surface tracking is at the heart of many modern Monte Carlo neutron transport applications, including MCNP \cite{MCNP_RisingArmstrongEtAl}, Shift \cite{hamilton_continuous-energy_2019, pandya_implementation_2016}, MONK/MCBEND \cite{richards_monk_2015}, and OpenMC \cite{romano_openmc_2015}.


\SetKwComment{Comment}{//}{ }
\DontPrintSemicolon
\begin{algorithm}

    $m =$ lookup material in current particle location

    $\Sigma_{t,m} =$ look up total macroscopic cross section of material $m$ 

    \While{particle is alive}{
    
        $d_{\text{collision}} = -\ln{\xi} / \Sigma_{t,m} $
        
        $d_{\text{surface}} =$ compute distance to nearest surface along particle direction of travel

        \If {$d_{\text{collision}} < d_{\text{surface}}$}{

                $d = d_{\text{collision}}$
                
                sample collision type
                
                carry out collision
                
            } \Else {

                $d = d_{\text{surface}}$
        
                move particle to surface
    
                $m =$ lookup material on the other side of the surface
    
                $\Sigma_{t,m} =$ look up total macroscopic cross section of material $m$ 
    
                \If {surface is a boundary}{
    
                    implement boundary condition
    
            }
        }
        score track-lengths to tally bins
    }   
    
    \vspace{1.5em}
    \caption{A generic surface tracking algorithm.}
    \label{alg:surface}
\end{algorithm}


Delta tracking is the next most common tracking approach and is shown in algorithm \ref{alg:trad}.
It starts by pre-processing a \textit{majorant} macroscopic cross-section
\begin{equation}
    \maj(E) = \max\left({\Sigma_{t,1}(E), \Sigma_{t,2}(E) \dots \Sigma_{t,m}(E) \dots \Sigma_{t,M}(E)}\right)
\end{equation}
such that it is the largest cross section at any given point in any material in the problem.
The algorithm to compute the majorant cross-section for nuclides on a non-unified energy grid currently implemented in MC/DC is in appendix \ref{app:majorant}.
Figure \ref{fig:majorant_c5ce} at right shows a macroscopic majorant cross section for a typical pressurized water reactor.
When delta tracking the distance to collision is always computed with the majorant,
\begin{equation}
    _{\text{collision}} = \frac{-\ln{\xi}}{\maj} \; ,
\end{equation}
ignoring surface crossing events.
However, this sampling forces extra collisions that do not physically exist; using the majorant generates the smallest distance to collision. Delta tracking algorithms use a rejection sampling to determine if the sampled collision event was \textit{real} or a \textit{virtual} (\textit{phantom}) collision.
At the particle's current location, the true material region of the is be identified to compute $\Sigma_{t,m}$.
If 
\begin{equation}
    \xi > \frac{\Sigma_{t,m}}{\maj} \; ,
\end{equation}
where $\xi$ is a new random number, the collision did not physically occur, is rejected, and the particle can be left alive on it's current direction of travel and energy.
From here the process is the same as before: tallies are accumulated, real collision physics are carried out when appropriate, and the algorithm continues as long as the particle is still alive.

\begin{figure}
    \centering
    \includegraphics[width=0.75\textwidth]{monte_carlo/delta_tracking/figures/macro_majorant_c5ce.pdf}
    \caption{Continuous energy macroscopic majorant cross-section ($\maj$) for the continuous energy version of C5G7 PWR described in appendix \ref{app:c5ce_mat}.}
    \label{fig:majorant_c5ce}
\end{figure}

When undergoing delta tracking some way to kill or reflect particles on vacuum or reflecting boundary conditions respectively is needed.
In a code that already implements surface tracking, the functionality to track distances to specific surfaces will already exist.
So it is natural when undergoing delta tracking to only compare distances to boundary surfaces exclusively.
Boundary surfaces are often planar, making the computation trivial and cheap.
%This is similar to other methods currently under development in OpenMC.

\begin{algorithm}

    \While{particle is alive}{
    
        $d_{\text{collision}} = -\ln{\xi} / \maj $
        
        $d_{\text{boundary}} =$ compute distance to boundary

        \If {$d_{\text{collision}} < d_{\text{boundary}}$}{

            $m =$ lookup material in current particle location
            
            $\Sigma_{t} =$ look up total macroscopic cross section of material $m$ 

            \If { $\xi < \Sigma_{t} / \maj$ }{

                collision is rejected

            } \Else {
            
                collision is accepted

                tally $1/\Sigma_t$ to bin at particles current location

                determine collision type 
                
                carry out collision
                
            }
                
        } \Else {
        
            move particle to boundary

            implement boundary condition
        }
        
    }
    
    \vspace{1.5em}
    \caption{Woodcock-delta tracking in MC/DC. Notably we are still surface tracking to boundaries.}
    \label{alg:trad}
\end{algorithm}

An added complication when using the Woodcock-delta tracking method is restricting what type of tallies can be efficiently scored.
There are many so called estimators that can be used to indicate quantities of interest (often scalar flux) by tallying events that occur within a given region of phase space.
The two common estimators are the collision estimator and the the track (or path) length estimator.
Often tallies are scored to bins on a structured mesh grid overlying the surfaces and material regions that a Monte Carlo simulations uses to conduct actual transport operations.
The track-length estimator is,
\begin{equation}
    \label{eq:pathlength}
    \hat{\phi}_n = \sum_{i=1}^{I}p_i \; ,
\end{equation}
where $\hat{\phi}_n$ is the integrated scalar flux over given mesh cell $p$ is the track-length of particle $i$ passing through mesh cell $n$.
The collision estimator is,
\begin{equation}
    \label{eq:collision}
    \hat{\phi}_n = \sum_{i=i}^{I} \frac{1}{\Sigma_{t,n}} \;.
\end{equation}
The collision estimator will often produce a more variant solution as compared to other estimators for tallies where $\Sigma_{t,m}$ is small (optically thin, less dense materials) and will never tally anything into a void region.
On the other hand the track-length estimator will always tally into every mesh bin as particles moves.
Meaning, undermost problem regimes, more information will be scored, which will result in a lower-variant tally for the same number of particles.

Both of the estimators in Eqs. \ref{eq:collision} and \ref{eq:pathlength} are only for flux integrals, where response functions are one.
Other response functions for other reaction rates (e.g. fission rate density) will require the knowledge of the macroscopic cross section of that operation in a given location.
In this work we limit ourselves to flux integrals only.
Thus we also only enable these schemes for fixed source problems and leave k-eigenvalue calculations for future work.
This does currently limit the more general applicability of the methods we implement, tho future work may extend these methods to allow for the computation of these perimeters.
We discuss this further in section \ref{disucssions}.

Delta tracking algorithms are implemented in many production Monte Carlo neutron transport applications including Serpent \cite{leppanen_2010_burnup, leppanen_use_2017, leppanen_development_2013, leppanen_2015_serpent}, MONK/MCBEND \cite{richards_monk_2015}, and GUARDYAN \cite{molnar_gpu_based_2019}.
Notably, the MONK Monte Carlo neutron transport code is the direct successor to the GEM code where Woodcock et. al first implemented Woodcock-delta tracking \cite{woodcock_techniques_1965}.
There are other weighted Woodcock-delta tracking algorithms, in this work we explore variants of non-weighted version \cite{molnar_variance_2018, morgan_weighted-delta-tracking_2015}.

% what other codes do and how this work is novel
Modern transport applications often choose one-or-the other tracking algorithm and optimize from there. However, either method of sampling the probability distribution functions can hold valid for the same system in the same simulation even at the same location in phase space.
Serpent Monte Carlo code establishes regions of delta tracking and surface tracking based on the ratio between the ratio of $\Sigma_{t,m}$ and $\maj$.
MONK/MCBEND support material regions inside of which delta tracking is implemented (called Hole geometries) where surface tracking is implemented else where.
Ongoing developments in OpenMC and MCNP will introduce similar delta tracking algorithms for 
Publications indicate that no other Monte Carlo neutron transport application currently supports the use of any track-length estimator within a region undergoing Woodcock-delta tracking.


\section{Hybrid Delta Tracking Schemes}

In this section we introduce and verify the voxilized tally structure that allows MC/DC to use a track-length estimator while Woodcock-delta sampling.
We also introduce two hybrid surface-delta tracking schemes we implement in time-dependent transport with moving surfaces in MC/DC.
The first is a region based delta tracking method which has been implemented before in other production Monte Carlo neutron transport applications.
The second and novel method where delta tracking is used in high-energies above neutron cross-section resonances, where total cross sections are often similar to the majorant, and track-lengths are long relative to nominal system dimensions.


\subsection{Voxelized Tallies}

Traditional Woodcock-delta tracking algorithms do not keep track of what material or physical mesh cell a given particle occupies at any moment in transport.
Conventional wisdom dictates that repeatedly conducting lookups of locations, cross sections, and tally bin indices while delta-tracking would be prohibitively costly and eliminate any boon to performance that system homogenization gave.
It follows that Woodcock-delta tracking precludes use of a track-length estimator for quantities of interest.
Crucially there is nothing \textit{mathematically} preventing the use of a track-length estimator with Woodcock-delta tracking, only engineering issues.
Furthermore if all that are needed to quantify a given problem is the total integral flux no such extra cross section lookups are needed.
Only an identification of the track-length and bin to accumulate.
This forms the underlying assumptions that are used in this work.

\begin{figure}[!htb]
  \centering
  \includegraphics[width=\textwidth]{monte_carlo/delta_tracking/figures/tally_ray.pdf}
  \caption{A particle tallying to multiple tally mesh bins (shown in red). Implemented as a single operation for both surface and Woodcock-delta tracking in MC/DC.}
  \label{fig:tally_ray}
\end{figure}

MC/DC v0.11.1 \cite{morgan_monte_2024} has an optimized tally algorithm where track-lengths to multiple structured mesh cells (which we call voxels
\footnote{\textit{voxel} is the 3D analog to a pixel, here we mean it as a cube mesh bin to tally into}
)are scored in a single operation.
This algorithm is based on a sweeping method where the initial state (mesh cell index, exact ($x$, $y$, $z$) position, direction of travel, speed, and particle clock) are known, as is the distance to the next event.
The particle is then swept from voxel to voxel, tallying the exact track-length traveled in a given voxel along the way.
Figure \ref{fig:tally_ray} shows a hypothetical particle track and the mesh bins to be tallied to (in red) in a single operation.
In MC/DC's normal algorithm this is done even before moving a particle to an event while surface tracking.

In this work, we used this voxlized tally scheme when undergoing Woodcock-delta tracking.
Even if a particle collision is rejected that particle still physically moved to the location that was sampled, allowing the use of the track-length estimator.
In this scheme the distance tallied is always the distance sampled with the majorant, baring census crossings or distance to boundary events.

To verify that MC/DC's voxelized track tallies with delta tracking both converges to the correct solution and at the correct rate we use four anaclitic based benchmark problems.
We compare the error from integral quantities of interest to a reference solution then plot the error as a function of increasing particle count.
The convergence rate should be the standard Monte Carlo convergence rate of $N^{-1/2}$.
Our verification problems are
\begin{itemize}
    \item AZURV1 time dependent benchmark both super and sub critical (figure \ref{fig:azurv1}) \cite{ganapol_homogeneous_2001};
    \item A time dependent infinite pin cell using the 371 group SHEM cross sections (figure \ref{fig:shem}) \cite{hfaiedh_2005_shem}; and 
    \item Reed's Problem (figure \ref{fig:reeds}) \cite{reed_difference_1971};
    \item A purely absorbing slab (figure \ref{fig:abs_slab}).
\end{itemize}
\begin{figure}
  \centering
  \includegraphics[width=0.32\linewidth]{monte_carlo/delta_tracking/figures/verification/azurv1/azurv1_flux.png}
  \includegraphics[width=0.32\linewidth]{monte_carlo/delta_tracking/figures/verification/azurv1/azurv1_census_flux.png}
  \includegraphics[width=0.32\linewidth]{monte_carlo/delta_tracking/figures/verification/azurv1/azurv1_census_tally_flux.png}
  \includegraphics[width=0.32\linewidth]{monte_carlo/delta_tracking/figures/verification/azurv1/azurv1_sub_flux.png}
  \includegraphics[width=0.32\linewidth]{monte_carlo/delta_tracking/figures/verification/azurv1/azurv1_super_flux.png}
  \caption{Convergence rate verification of AZURV1 \cite{ganapol_homogeneous_2001} }
  \label{fig:azurv1}
\end{figure}
\begin{figure}
  \centering
  \includegraphics[width=0.32\linewidth]{monte_carlo/delta_tracking/figures/verification/shem/inf_shem361_flux.png}
  \includegraphics[width=0.32\linewidth]{monte_carlo/delta_tracking/figures/verification/shem/inf_shem361_td_n.png}
  \includegraphics[width=0.32\linewidth]{monte_carlo/delta_tracking/figures/verification/shem/inf_shem361_td_census_n.png}
  \includegraphics[width=0.32\linewidth]{monte_carlo/delta_tracking/figures/verification/shem/inf_shem361_td_flux.png}
  \includegraphics[width=0.32\linewidth]{monte_carlo/delta_tracking/figures/verification/shem/inf_shem361_td_census_flux.png}
  \caption{Convergence rate of an infinite pin using the SHEM 361 group cross section library \cite{hfaiedh_2005_shem}}
  \label{fig:shem}
\end{figure}
\begin{figure}
  \centering
  \includegraphics[scale=0.75]{monte_carlo/delta_tracking/figures/verification/reed/reed_flux.png}
  \caption{Convergence rate of flux from Reed's problem \cite{reed_difference_1971}}
  \label{fig:reeds}
\end{figure}
\begin{figure}
    \centering
    \includegraphics[width=0.75\linewidth]{monte_carlo/delta_tracking/figures/verification/abs_slab/slab_absorbium_flux.png}
    \caption{Convergence rate of QOIs for a purely absorbing slab wall problem}
    \label{fig:abs_slab}
\end{figure}

All verification simulations show the $N^{-1/2}$ convergence rate expected for Monte Carlo results.
This verifies that we are getting the expected results with voxelized Woodcock-delta tracking.


\subsection{Hybrid-In-Material}
\label{sec:material_exc}


The first hybrid method we implement in MC/DC is a material or region based decision on weather to us surface or delta tracking we call ``hybrid-in-material".
Each particle is given an additional flag declaring which transport algorithm it is using to sample a distance to next event.
At the beginning of the \texttt{determine\_next\_event} function the tracking flag will be set to \texttt{true} if the particle is in a material region declared by the user to undergo delta tracking or \texttt{false} if in a region where delta tracking should not be used. 
This is similar to in Serpent2's algorithm however our voxelized tally structure allows us to use track-length estimators in all regions not just those undergoing surface tracking.
Also Serpent2's algorithm makes automatic decisions about where to surface and delta track based off a user supplied cut off value \cite{leppanen_development_2013} whereas we leave it as user option for now.


\begin{figure}
    \centering
    \includegraphics[width=0.75\textwidth]{monte_carlo/delta_tracking/figures/macro_majorant_dragon.pdf}
    \caption{Continuous energy macroscopic majorant cross-section ($\maj$) for the Dragon burst problem.}
    \label{fig:majorant_dragon}
\end{figure}

Conventional wisdom dictates that delta tracking should not be used in systems with strong localized absorbers as the majorant will be governed by cross sections much larger then others in the system.
Figure \ref{fig:majorant_dragon} shows the cross section information for the Dragon burst problem we describe in section \ref{sec:benchmarks}.
Conventional wisdom would suggest that this is a simulation that does not warrant Woodcock-delta tracking.
This simulation contains three materials, two of which (the fuel and tamper) have similar cross sections over all energies clumped together.
However the cross sections modeling air are over four orders of magnitude under cross sections describing the fuel and tamper.
This means in the air, particles will get stuck in the rejection sampling loop having phantom collision after phantom collision and statistically rarely complete a particle history.
Furthermore for classical delta tracking (where only a collision estimator can be used) the tallies in the air will be highly variant as there will be statistically few collisions taking place.
Using surface tracking in the air and delta tracking in the material may give a performance boost in this problem.

\subsection{Hybrid-In-Energy}
\label{sec:cutoff}

The second hybrid method we call ``hybrid-in-energy", and is a decision to do delta tracking above a given user-defined energy and conduct surface tracking below.
For neutrons, high energies (or speeds) are characterized by relatively long mean free paths (small cross sections) as they are physically going too fast to interact with anything.
Delta tracking generally does better under these conditions compared to surface tracking as surface tracking would get stuck moving particles from region to region whereas delta tracking can stream particles through the whole problem.
Furthermore at higher energies in many systems the majorant will more closely match the cross sections of materials in alleviating issues with the rejection sampling loop.

For example consider a continuous energy version of the C5G7 benchmark reactor geometry \cite{jia_hou_oecdnea_2017}.
The material composition is given by table \ref{tab:c5ce} in appendix \ref{app:c5ce_mat}.
Figure \ref{fig:majorant_c5ce} shows the macroscopic material cross-sections for the 7 material reactor with the majorant in black.
Around \SI{10}{\kilo\electronvolt} the neutron resonances end and the problem can be considered high energy.

If using delta tracking for this whole problem neutrons will get stuck in resonance frequencies where the rejection sampling loop will be called too much degrading the performance of delta tracking.
Therefore it would be ideal to delta track above \SI{50}{\kilo\electronvolt} (as to completely avoid neutron resonances) and surface track under that threshold.

\subsection{Implementation in MC/DC}
\label{sec:implementation}

Monte Carlo Dynamic Code is purpose built to implement and test time dependent Monte Carlo novel numerical methods at scale \cite{morgan_2025_monte}.
It uses a novel for the field development structure where Python scripted compute kernels are compiled via the Numba compiler to run on CPUs and with the Harmonize GPU runtime manager to run on GPUs.

MC/DC allowed us to rapidly experiment with these methods at scale on both CPUs and GPUs on time dependent problems of interest.
Delta tracking methods can be easy to implement in a code that already conducts surface tracking. 
Generally to implement delta tracking in a code that already implements surface tracking add are:
\begin{enumerate}
    \item Pre-process functions to generate a majorant (MC/DC's implementation is in \ref{app:majorant});
    \item \texttt{if} statements in \texttt{distance\_to\_next\_event} functions to compute relevant distances;
    \item Functions to implement boundary conditions when in delta tracking mode; and
    \item Elevate delta tracking options to the input deck.
\end{enumerate}
Our implementation added about \num{450} lines of code, of which half was to produce various types of majorants (an example can be found in \ref{app:majorant}). 
Only about \num{200} lines of code where required in the compute kernels themselves to implement delta tracking in MC/DC.
This process was similar to implementing hybrid surface-delta tracking methods in MCATK \cite{morgan2023delta}.
This is also similar to current ongoing work in OpenMC.

We found the most complicated issue when implementing delta tracking in MC/DC to be the boundary conditions.
Surface tracking codes (like MC/DC) often enforce boundary conditions as a surface options, something to avoid when delta tracking.
Vacuum boundary conditions are simple as if or when a particle is determined to be out of the problem when looking for a cross section in the rejection sample particles are killed.
For reflecting surfaces we use a stripped down version of \texttt{distance\_to\_nearest\_surface} function in MCDC to compute distances to reflecting surfaces, if any.
Many reflecting boundary conditions including the ones we implement in our benchmark problems are imposed on planar surfaces, making the distance to boundary computation cheap.
In effect we are still surface tracking only to our reflecting boundary surfaces.


\section{Verification of Woodcock-Delta Tracking with Continuously Moving Surfaces}

To verify that Woodcock-delta tracking can be used in conjunction with the continuously moving surfaces in MC/DC we use the moving pellet regression test from MC/DC's test suite \cite{morgan_monte_2024}.
In it a rectangular fuel element which moves through a region with a small source.
As the pellet moves closer and farther from the source region the fission rate in the pellet changes.
Figure \ref{fig:moving_pellet} at left shows the fission reaction rate density at various points in time.
The outline of the pellet can be seen clearly.
These plots where produced using Woodcock delta tracking with a collision estimator and matched plots produced when using surface tracking with a collision estimator.

\begin{figure}
    \centering
    \includegraphics[width=0.49\linewidth]{monte_carlo/delta_tracking/figures/verification/moving_pellet_plot.pdf}
    \includegraphics[width=0.49\linewidth]{monte_carlo/delta_tracking/figures/verification/moving_pellet.pdf}
    \caption{(left) Fission flux density at various points in time for the moving pellet problem, using Woodcock-delta tracking with a collision estimator (right) convergence between fluxes produced from surface and Woodcock-delta tracking both with the track-length estimator, showing $N^{-1/2}$ convergence rate.}
    \label{fig:moving_pellet}
\end{figure}

To further verify Woodcock delta tracking with continuous movement physics we regressively compare the flux solutions provided from traditional surface tracking and delta tracking with voxelized tallies at various particle counts.
We compute 
\begin{equation}
    \epsilon_N = |\phi_N^{\text{surface}} - \phi_N^{\text{delta}} |_2
\end{equation}
at every choice of $N$ particles.
We then compare the error of flux over particles to ensure the expected Monte Carlo convergence rate ($N^{-0.5}$).
Figure \ref{fig:moving_pellet} at right shows the error converging at the expected rate.
This regressively verifies that Woodcock-delta tracking can be used in conjunction with continuously moving surfaces.
We also produced this same plot for both tracking method using a collision estimator for both flux and fission tallies all of which matched the expect Monte Carlo convergence rate.

\section{Benchmark Problems}
\label{sec:benchmarks}

% vairiance reudciton and figure of merrit
The performance of a given Monte Carlo algorithm for a specified problem is a function of solver variance ($\sigma^2$, from the Monte Carlo process itself) and wall-clock runtime it takes to get that solution.
If a certain algorithm can forum a solution with low variance at fewer particles it may still yet be not considered as suitable as an algorithm that takes many many more particles to reach that statistical variance if it is too slow.
So a measurement must be used to take into account both the variance of a given solution and Figure of Merit (FOM) is one such measure. In this work we will use 
\begin{equation}
    FOM = \frac{1}{\hat{\sigma}^2 t_{wc}}
\end{equation}
where $\hat{\sigma}^2$ is the L$_{1}$ norm (over phase space) of the variance of the solution provided by the Monte Carlo solver and $t_{wc}$ is the wall clock runtime required to obtain that solution.

\begin{figure}
    \centering
    \includegraphics[width=0.48\linewidth]{monte_carlo/delta_tracking/figures/dragon.png}
    \includegraphics[width=0.48\linewidth]{monte_carlo/delta_tracking/figures/c5g7.png}
    \caption{Schematics (left) Dragon burst problem \cite{kimpland2021dragon} (right) C5G7 reactor quarter (via reflecting boundaries) \cite{jia_hou_oecdnea_2017}}
    \label{fig:schems}
\end{figure}

% the problems we run themselves
We use four fully time dependent fixed source benchmark problems to compare the voxlized tally scheme and hybrid methods proposed in this work to surface tracking on CPU and GPU machines.
Two are multi-group, two are continuous energy.
Table \ref{table:benchmark_problems} summarizes the size (in both mesh and particle count) of each benchmark problem.
Some problems may not be suited for delta tracking methods but are used anyway here to show both correctness under physical problem dynamics (i.e., moving surface) and demonstrate algorithmic improvement in under highly exaggerated problem parameters.

\begin{table}[htb]
  \centering
  \begin{tabular}{@{}l c c c @{}} \toprule
    Problem & $N_{mesh}$ & $N_{particles}$ & Energy physics \\ \midrule
    Kobyashi & \num{120e3} & \num{1e10} & MG (1 group) \\
    Dragon Burst & \num{4.0e6} & \num{3e9} & Cont. E (3 materials) \\
    C5G7 & \num{3.9e6} & \num{1e7} & MG (7 groups) \\
    C5CE & \num{544e3} & \num{1e5} & Cont. E (7 materials) \\
    \bottomrule
  \end{tabular}
  \caption{Time dependent benchmark problems.} 
  \label{table:benchmark_problems} 
\end{table}

% koby intro
The first benchmark we consider is the time dependent version of the Kobyashi problem \cite{Kobayashi2001} introduced by \cite{morgan_2025_monte}.
We run this problem with 10 batches with \num{1e9} particles per batch (\num{1e10} particles total) with traditional surface tracking, traditional delta tracking with a collision estimator, and a delta tracking using the voxelized tally method.
This problem contains two materials, a dense region (characterizing a solid), and a low density region modeled by air (characterizing a void) in a single energy group.
There are a 120k structured tally mesh cells in the $x$-$y$ plane and in time.

% acident intro
Figure \ref{fig:schems} at right shows the geometry for the next two simulations we consider based on the C5G7 benchmark problem \cite{jia_hou_oecdnea_2017}.
We model a 4 phase accident where-in a pressurized light water reactor reactor is powering up by removing control rods.
Figure \ref{fig:c5g7} at left shows the normalized flux density as a function of time (produced from reactor point kinetics) and the four phases shaded as gray, green, red, and blue.
Phase one (shaded gray) starts with the reactor operating at steady state.
In phase two (shaded green) the control rods are removed from the reactor to power up to a new steady state.
Phase three (shaded red) begins when a bank of control rods gets stuck in the fully withdrawn position.
Towards the end of phase three the reactor sees a rapid spike in power that ends when at \SI{15}{\s} all the control rods are forced into the reactor ending the accident.
The fourth and final phase sees the reactor decaying in power as the delayed neutron population dies out.
Figure \ref{fig:c5g7} shows plots of flux in the $x$-$y$ plan at various points in time including at \SI{14.95}{\s} the maximum power excursion.
These results are produced from running \num{1e6} particles in \num{10} batches (total of \num{1e7} particles).

\begin{figure}
    \centering
    \includegraphics[width=0.48\linewidth]{monte_carlo/delta_tracking/figures/c5/acc.pdf}
    \includegraphics[width=0.48\linewidth]{monte_carlo/delta_tracking/figures/c5/flux.pdf}
    \caption{C5G7 stuck rod accident simulation, (left) flux densities through time, (right) scalar flux on x-y plane (top view) at points in time}
    \label{fig:c5g7}
\end{figure}

We first model this problem using the 7 group materials described by the normal C5G7 benchmark \cite{jia_hou_oecdnea_2017} which we will call C5G7 in the remainder of this work.
We model C5G7 with \num{3.9} million mesh cells in a full 3D, time and energy dependent tally mesh.
The movement of the control rods into and out of the reactor is controlled with MC/DC's continuous movement functionality.
To verify that the delta tracking methods we explore converge to the correct solutions while in transport with continuously moving surfaces we compare delta tracking solutions to solutions provided by surface tracking (which has already been verified and validated).
Performance data is collected by running \num{1e6} particles in \num{10} batches (total of \num{1e7} particles).

Next we define a continuous energy version of the C5G7 geometry we call C5CE undergoing the same 4-phase accident.
Table \ref{tab:c5ce} in appendix \ref{app:c5ce_mat} shows material compositions for the C5CE problem.
Using this benchmark we evaluate both the voxelized tally method as well as the hybrid-in-energy method described in section \ref{sec:cutoff}.
Figure \ref{fig:majorant_c5ce} at right plots the macroscopic total and majorant cross sections over energy for the materials in C5CE.
The neutron resonances end around \SI{10}{\kilo\electronvolt} so we set a cut off at \SI{50}{\kilo\electronvolt}.
When running this problem particles moving at speeds above \SI{50}{\kilo\electronvolt} MC/DC will use voxelized Woodcock-delta tracking and blow it will use traditional surface tracking.
We expect this to provide significant speedup over all previous methods explored in this problem as we avoid both down falls of either tracking method (surface tracking moving from surface to surface and delta tracking getting stuck in the rejection sample). 
To quote Miley Cyrus, ``you get the best of both worlds" \cite{cyrus_best_2005}.
C5CE also includes the same continuously moving surfaces as C5G7 so it serves as an additional verification that delta tracking methods can be used in conjunction continuously moving surfaces.
We model this problem with \num{447} mesh tally bins in 2D $x$-$y$ geometry (integrated along $z$) time and energy dependent tally mesh.
We run \num{1e5} particles in a single batch.

% dragon intro
The next problem we consider is a full time dependent simulation of the historical Dragon-burst experiments \cite{kimpland2021dragon}.
It was conducted in 1944 during the Manhattan project to prove that criticality could be achieved with prompt neutrons only.
Previous experiments, like the Chicago Pile One, used delayed neutrons to achieve criticality.
Figure \ref{fig:schems} at left shows a schematic of the experiment where a slug of highly enriched (75\%) Uranium Hydride ($UH_{10}$) was doped through a core with additional fuel.
As the slug moved through the core a prompt critical reaction was triggered as critical mass was achieved.
Before the super critical burst could become a problem (i.e. a deadly uncontrollable blast), gravity would pull the slug out of the core, ending the reaction.
Kimpland et. al showed that this burst criticality experiment could achieve over 9 orders of burst magnitude \cite{kimpland2021dragon}.
Work is on going to model the problem up to 9 orders of burst magnitude in MC/DC but here we use a less reactive version (25\% enrichment) to verify delta tracking with the hybrid-in-material method described in section \ref{sec:material_exc} and additional verification for delta tracking with continuously moving surfaces.
Figure \ref{fig:dragon_results} shows the overall flux density through time at left and a y-z plot of scalar flux at various points in time on the right.

\begin{figure}
    \centering
    \includegraphics[width=0.48\linewidth]{monte_carlo/delta_tracking/figures/dragon/dragon_curve.pdf}
    \includegraphics[width=0.48\linewidth]{monte_carlo/delta_tracking/figures/dragon/flux_dragon.pdf}
    \caption{Dragon burst simulation, (left) flux density through time, (right) scalar flux on y-z plane (side view) at points in time.}
    \label{fig:dragon_results}
\end{figure}

Figure \ref{fig:majorant_dragon} at left shows the continuous energy macroscopic total cross sections in the model we simulate.
The majorant, tamper and fuel cross sections all sit almost 4 orders magnitude above the cross section for air.
This is due to air being low density meaning fewer atoms for neutrons to interact with.
We expect any delta tracking algorithm to perform quite poorly in this model for a number of reasons.
First when undergoing delta tracking particles in the air region will get stuck in the rejection sample loop for a long time as the majorant is orders of magnitude larger and will sample very small distance before conducting a rejection sample.
Second as much of the problem is a void region the collision estimator that normal delta tracking requires will provide poor tallies in those regions
Third the problem is geometrically simple: consisting of a rectangular slug moving through a rectangular slab with a complimentary hole.
Traditional wisdom suggests that delta tracking performs best in complex materials.
The point of comparing delta tracking methods for this specific problem is to first confirm that delta tracking works with such a dynamic problem (the slug is modeled using MC/DC's continues movement functionality) and if we measure superior performance to standard delta tracking when using the hybrid-in-material method.

\section{Results}

% TLEase add the following required packages to your document preamble:
% \usepackage{multirow}
\begin{table}
\centering
\begin{tabular}{llllll}
\hline
Problem & Tracking Alg. & Estimator & Runtime [s] & $|\hat{\sigma}^2|_1$ & FOM \\ \hline
\multirow{4}{*}{Kobyashi}
 & surface  & TLE & \num{1574} & \num{4.514e-3} & 0.1407 \\
 & surface  & CE & \num{1014} & \num{9.552e-2} & 0.0103 \\
 & delta  & TLE & \num{1298} & \num{4.539e-3} & 0.1697 \\ 
 & delta  & CE & \num{817.2} & \num{2.062e-1}  & 0.0059 \\
 \hline
 

 
\multirow{4}{*}{C5G7}
 & surface  & TLE & \num{7926} & \num{1.295e-4} & \num{0.9744} \\
 & surface  & CE & \num{3173} & \num{1.295e-4} & \num{2.4336} \\
 & delta  & TLE & \num{2870.} & \num{1.400e-4} & \num{2.4889} \\
 & delta  & CE & \num{1820} & \num{2.15e-4} & \num{2.5515} \\
 \hline

 
\multirow{6}{*}{C5CE} 
 & surface  & TLE & \num{734.6} & \num{1.812e-3} & \num{0.7484}\\
 & surface  & CE & \num{587.8} & \num{1.812e-2} & \num{0.9391}\\
 & delta  & TLE & \num{496.8} & \num{4.906e-3} & \num{0.4103} \\
 & delta  & CE & \num{555.0} & \num{1.173e-2} &  \num{0.1536} \\
 & hybrid-energy & TLE & \num{226.25} & \num{1.164e-3} & \num{3.798} \\ 
 & hybrid-energy & CE & \num{220.5} & \num{1.144e-3} & \num{3.965} \\ 
 \hline

 
\multirow{4}{*}{Dragon} 
 & surface  & TLE & \num{3816} & \num{1.163e-6} & \num{221.4} \\
 & delta  & CE & DNF$^*$ & - & - \\
 & delta  & TLE & DNF$^*$ & - & - \\
 & hybrid-in-material & TLE & \num{15493} & \num{1.106e-6} & \num{58.40} \\
 \hline
\end{tabular}
\caption{Results for benchmark problems on Dane (112$\times$ Intel Sapphire Rapids CPU cores). $^*$Did not finish in 8 hour time limit.}
\label{tab:dane_results}
\end{table}

%1.163327654958653e-06
%1.1056932706381267e-06

% Please add the following required packages to your document preamble:
% \usepackage{multirow}
\begin{table}
\centering
\begin{tabular}{llllll}
\hline
Problem & Tracking Alg. & Estimator & Runtime [s] & $|\hat{\sigma}^2|_1$ & FOM \\ \hline
\multirow{4}{*}{Kobyashi} 
 & surface  & TLE & \num{973.9} & \num{4.514e-3} & 0.2275 \\
 & surface  & CE & \num{840.6} & \num{9.952e-2} & 0.0125 \\
 & delta  & TLE & \num{831.8} & \num{4.539e-3} & 0.1697 \\ 
 & delta  & CE & \num{620.7} & \num{2.062e-1}  & 0.0078 \\
 \hline
 
\multirow{4}{*}{C5G7} 
 & surface  & TLE & \num{4598} & \num{1.305e-4} & \num{1.666} \\
 & surface  & CE & \num{2403} & \num{1.305e-4} & \num{3.1878} \\
 & delta  & TLE & \num{161} & \num{1.400e-4} & \num{4.4205} \\ 
 & delta  & CE & \num{789.3} & \num{2.152e-4} & \num{5.8860} \\

 \hline
 
\multirow{6}{*}{C5CE} 
 & surface  & TLE & \num{550.9} & \num{2.452e-3} & \num{0.7402}\\
 & surface  & CE & \num{465.9} & \num{2.452e-3} & \num{0.8753}\\
 & delta  & TLE & \num{500.7} & \num{1.690e-2} & \num{0.1182} \\
 & delta  & CE & \num{421.6} & \num{1.793e-2} &  \num{0.1323} \\
 & hybrid-energy & TLE & \num{291.7} & \num{1.557e-3} & \num{2.2023} \\ 
 & hybrid-energy & CE & \num{275.6} & \num{1.839e-3} & \num{1.9731} \\ 
 \hline
 
\multirow{4}{*}{Dragon} 
 & surface & TLE & \num{3993} & \num{1.163e-06} & \num{2153} \\
 & delta & CE & DNF$^*$ & - & - \\
 & delta & TLE & DNF$^*$ & - & - \\
 & hybrid-material & TLE & DNF$^*$ & - & - \\ \hline
\end{tabular}
\caption{Results for benchmark problems on Lassen (4$\times$ Nvidia Tesla V100). $^*$Did not finish in 8 hour time limit.}
\label{tab:lassen_results}
\end{table}

In this section we discuss performance results of the of the benchmarks described above with surface, traditional delta, voxelized delta tracking as well as the hybrid-in-energy and hybrid-in-material tracking algorithms.
In this section we also verify that the various delta tracking methods converge to the same solution as surface tracking while transporting on a geometry with continuously moving surfaces.

\subsection{Performance Results}

% testing systems
We ran our benchmark problems on high-performance computing systems available at Lawrence Livermore National Laboratory (LLNL): the Dane, and Lassen machines.
Dane is a CPU-only system with dual-socket Intel Xeon Sapphire Rapids CPUs, each with 56 cores for a total of 112 per node. 
Lassen is the open collaboration sibling to the Sierra machine with four Nvidia Tesla V100s GPUs and two IBM Power 9 CPUs per node.
We will make all performance statements against CPU runtime from a whole node of Dane (112 MPI threads, CPU) and a whole node of Lassen (4 MPI Threads, GPU).
We compile to Nvidia GPUs with CUDA v11.8 and
Nvida-PTX with Numba v0.59.0\footnote{Numba v0.59.1 is the most recent version to support Power9 CPUs}.
We compile to CPUs with Numba v0.60.0.
We build our delta tracking methods off of MC/DC v0.12.0 \cite{transport_cement_mcdc_2024} and compile with Harmonize v0.0.2 \cite{harmonize}.
We use double precision for all floating point operations.
% Introduce the table
Table \ref{tab:dane_results} and \ref{tab:lassen_results} show the wall-clock runtime, normalized standard deviations and figures of merit for our four benchmark problems from the Dane and Lassen machines respectively.

% Results Koby
The Kobyashi problem on Dane sees dramatic runtime improvements when moving from surface to delta tracking with a collision estimator.
However this does not make up for the two orders of magnitude additional variance on the tallies of interest.
This leads to a dramatically smaller figure of merit for traditional delta tracking (using a collision estimator) compared to surface tracking.
Figure \ref{fig:koby} shows the traditional delta tracking with a collision estimator on the bottom and our voxlized delta tracking on top at various points in time through the simulation.
It is clear to a casual observer that the collision estimator is producing a more variant (appearing as static-y or fuzzy) result than the track-length estimator.
Our voxlized tally method sees 21\% decrease in wall clock time compared with the same amount of normalized variance leading to a moderately improved figure of merit (\num{0.1697} and \num{0.1407} for our voxelized method and surface tracking respectively) 
This pattern of results, where traditional delta tracking is the fastest wall clock runtime with a normalized variance orders of magnitude above methods using a track-length estimator, will persist through the analysis of the other problems we consider.
As will the behavior of the voxelized delta method's runtime sitting between surface tracking and traditional delta tracking with similar variance to surface tracing results.


\begin{figure}
    \centering
    \includegraphics[width=\textwidth]{monte_carlo/delta_tracking/figures/cle_v_tle.pdf}
    \caption{Comparison of delta tracking using the track-length estimator (top row) and collision estimator (bottom row) at three points in time. Collision estimator has a much higher variance.}
    \label{fig:koby}
\end{figure}

% Restuls C5G7
Moving on to C5G7, the same pattern of results persists where traditional delta tracking sees the smallest wall-clock runtime with the highest variance (\SI{1673}{\s} and \num{0.20062} respectively), while surface tracking sees the longest runtime and voxelized delta tracking sits between the two (\SI{7920}{\s}, \SI{2870}{\s} respectively) with roughly the same error ($\approx$\num{1e-4}).
The pattern continues to persist on Lassen only with between \num{1.7}$\times$ and \num{2}$\times$ wall clock runtime speedup for all tracking methods when moving from Dane to Lassen while variance remains about the same.
Voxelized delta tracking shows a \num{2.6}$\times$ and \num{2.7}$\times$ higher figure of merit over surface tracking on Dane and Lassen respectively.

% Restuls C5CE
C5CE shows similar results to the previous benchmarks for surface traditional and voxelized tracking methods.
The speedup of Lassen over Dane is now lower (between \num{0.9}$\times$ and \num{1.3}$\times$ speedup) indicating improvements are needed for our continuous energy physics when implemented on GPU.
The hybrid-in-energy method sees the most significant improvement of figure of merit with an order of magnitude increase (\num{11}$\times$ on Dane \num{7}$\times$ on Lassen).


% Restuls Dragon
The performance of the Dragon problem is an outlier in the behavior of the methods we consider.
Delta tracking (both traditional and voxelized) do not finish on Dane or Lassen in the 8 hours given to complete the simulation on either machine.
As predicted the hybrid-in-material method does see improvement over the other two methods, as it completes the simulation in 4 hours while traditional surface tracking completes it in one.
Also as predicted this is a problem that does not warrant any delta tracking methods due to the material composition and geometric layout of the problem.
However, It further demonstrates that delta tracking methods can be used with the continuously moving surfaces implemented in MC/DC.


\section{Discussion, Conclusions, and Future Work}
\label{disucssions}

We have implemented, Woodcock-delta tracking in MC/DC on CPUs and GPUs including using a voxelized track-length tally to efficiently score integral track-length tallies while under going delta tracking.
We verify the solution produced by this method against various steady state and time dependent anaclitic benchmark problems available in MC/DC's verification suite.
We also verify the Woodcock-delta tracking algorithm with the continuously moving surfaces in MC/DC using the fuel pellet problem.
We have shown figures of merit improve modestly in large scale multi-group and continuous energy time dependent benchmark problems when using this tracking and tallying technique on CPUs and GPUs.

We have also demonstrated a novel hybrid surface-delta tracking scheme called hybrid-in-energy where surface and delta are used in low and high energies, respectively.
For a continuous energy version of the C5G7 benchmark geometry undergoing a 4-phase accident we show an order of magnitude increase in figure of merit on both a CPU and GPU node.
We have also confirmed previously defined surface-hybrid delta tracking methods can produce better results in problems significant void regions \cite{leppanen_2010_burnup}.

A combination of a numerical method getting a lower variance with fewer particles and an efficient implementation of that numerical method.
Thus what makes a delta or surface tracking algorithm more ``efficient" than the other can vary code-to-code.
Not every code implements tallying the way MC/DC does meaning the added efficiency of using the voxelized tally 
This does make the repeatability of surface v delta tracking implementations less generalizable but this has been the case when confronting multiple developments in the Monte Carlo neutron transport space.
For example when implementing GPUs some production codes had relatively few algorithmic issues and substanital CS

The major take away of this work is that Delta and surface tracking do not have to be treated as discrete choices in numerical method.
Indeed this matches what the developers of Serpent2, and MONK/MCBEND have known for years.
Greater performance can be achieved by mixing and matching the underlying transport methods given the detectable physics of a simulation and the relative strengths and weakness of a given transport application.
As we discussed in section \ref{sec:implementation} taking a surface transport code and implementing delta tracking is simple given a method of computing a macroscopic majorant cross section.
This process has been completed at least twice now in MCATK \cite{morgan2023delta}, and MC/DC with work ongoing in OpenMC and MCNP. % cite their branch

For the voxelized tally method specifically this work is promising, but the lack of non-integral tallies means limited applicability to physical problems of interest.
Work is on-going in producing efficient methods of computing relevant macroscopic cross-sections defined on a structured mesh while a particle is undergoing transport.
This process is simple for multi-group cross-sections where reaction rates can be determined entirely as a post process but work is ongoing to identify the most efficient method for continuous energy transport.

A method of reducing the dimensionality of the majorant is also under active research in order to have a smaller more efficient majorant cross section lookup in the sample distance to collision operations.
Implementing post process reaction tallies for other quantities of interest
Experiments with the collision flux estimator and methods of tallying into vacuum and low interaction rate are being considered.
Research is also ongoing into methods to tallying with multiple estimators onto the same mesh.
If only one estimator is used at a time then we avoid the need for complex covariance computations (e.g., those implemented in MCNP \cite{urbatsch_estimation_1995} \cite{MCNP_RisingArmstrongEtAl}).

Combinations of the Woodcock-delta and surface tracking algorithms proves to be a compiling field of research in Monte Carlo neutron transport.
The combination of these two tracking algorithms allows for greater significances on a broader set of problem physics where one scheme may out perform another.
Either method is a valid sampling of the cumulative probability distribution function at any point in transport.
Modest to significant improvements may occor when taking advantage of that fact.


\section*{Acknowledgments}
We thank Patrick Shirwise and Paul Ramano of Argonne National Laboratory, Mike Rising of Los Alamos National Laboratory, Jaakko Leppänen of VTT Technical Research Center of Finland, and Simon Richards of ANSWERS software for productive conversations.
We also thank the high performance computing staff at Lawrence Livermore National Laboratory for continued support using the Dane machine. 

This work was supported by the Center for Exascale Monte-Carlo Neutron Transport (CEMeNT) a PSAAP-III project funded by the Department of Energy, grant number: DE-NA003967.

\section*{Appendix}


\subsection*{Continuous Energy Macroscopic Majorant}
\label{app:majorant}

To compute a unified energy grid per material we combine the energy grids from all nuclide in a given material. 
Then we call \texttt{numpy.unique()} which returns a sorted array (form smallest to largest) with no repeating elements \cite{van_der_walt_numpy_2011}.
To compute a unified energy grid for the whole problem we do the same but with all the nuclides in the entire problem.

Computing a macroscopic majorant for nuclides that are not on a unified energy grid requires two levels of interpolation to put a given macroscopic total cross section on a unified energy grid.
First interpolation from each nuclides microscopic cross section onto a materials unified energy grid to compute a macroscopic total cross section.
Then a second interpolation from the material to the majorant's unified energy grid.
We use \texttt{scipy.interpolate.interp1d()} to interpolate from one energy grid to the next \cite{2020SciPy-NMeth:a}. 

The following code shows exactly how this is done in MC/DC:

\begin{minted}[mathescape, linenos]{python}
import scipy
import numpy

# unify the energy grids from all nuclides
majorant_energy_grid = np.array([])
for n in range(N_nuclide):
    nuclide = mcdc["nuclides"][n]
    majorant_energy_grid = np.append(majorant_energy_grid, nuclide["E_xs"])

# sort energy grid and eliminate duplicate points
majorant_energy_grid = np.unique(majorant_energy_grid)
majorant_xsec = np.zeros_like(majorant_energy_grid)

for m in range(N_material):

    material = mcdc["materials"][m]

    material_energy_grid = np.array([])

    # copmute a unified energy grid across all nuclides of a given material
    for n in range(material["N_nuclide"]):
        nuclide = mcdc["nuclides"][n]
        material_energy_grid = np.append(
            material_energy_grid, nuclide["E_xs"]
        )
    material_energy_grid = np.unique(material_energy_grid)
    MacroXS = np.zeros_like(material_energy_grid)

    # compute the macroscopic total cross section of a material on its unified
    # energy grid
    for n in range(material["N_nuclide"]):
        ID_nuclide = material["nuclide_IDs"][n]
        nuclide = mcdc["nuclides"][ID_nuclide]

        # Get nuclide density
        N = material["nuclide_densities"][n]

        # putting the microscopic cross-sections on the unifed
        # material energy grid
        total_micro_xsec_unified = scipy.interpolate.interp1d(
            nuclide["E_xs"], nuclide["ce_total"], bounds_error=False
        )
        total_micro_xsec_unified = total_micro_xsec_unified(
            material_energy_grid
        )

        # Accumulate
        MacroXS += N * total_micro_xsec_unified

    # puting the total macroscopic cross sections on on the majorant energy grid
    total_xsec_unified = scipy.interpolate.interp1d(
        material_energy_grid, MacroXS, bounds_error=False
    )
    total_xsec_unified = total_xsec_unified(majorant_energy_grid)

    # compares old majorant xsec and the currently evaluated unified xsec 
    # and picks the larger xsecs
    majorant_xsec = np.max((majorant_xsec, total_xsec_unified), axis=0)

\end{minted}

This process results in a large majorant cross section that is quite unwieldy.
Other more efficient algorithms exist to produce a just as accurate majorant with fewer points.
Delta tracking codes like Serpent avoid the need for any of this by having all nuclides on a unified energy grid \cite{leppanen_2015_serpent}.

\newpage

\subsection*{C5CE Material Definition}
\label{app:c5ce_mat}
\begin{longtable}{|l|l|l|}
\hline
Material                         & Nuclide & Atom fraction          \\ \hline
\endfirsthead
%
\endhead
%
\multirow{5}{*}{UO2 Fuel}        & O-16     & \num{0.04585265389377734}    \\ \cline{2-3} 
                                 & O-17     & \num{1.7419604031574338e-05} \\ \cline{2-3} 
                                 & O-18     & \num{9.19424166352541e-05}   \\ \cline{2-3} 
                                 & U-235    & \num{0.0007217486041189947}  \\ \cline{2-3} 
                                 & U-238    & \num{0.02224950230720295}    \\ \hline
                                 & O-17     & \num{1.743649552488715e-05}  \\ \cline{2-3} 
\multirow{5}{*}{MOX-43 Fuel}     & O-16     & \num{0.04589711643122753}    \\ \cline{2-3} 
                                 & O-17     & \num{1.743649552488715e-05}  \\ \cline{2-3} 
                                 & O-18     & \num{9.203157163056531e-05}  \\ \cline{2-3} 
                                 & U-235    & \num{0.0003750264168772414}  \\ \cline{2-3} 
                                 & U-238    & \num{0.02262319599228636}    \\ \hline
\multirow{5}{*}{MOX-7 Fuel}      & O-16     & \num{0.04583036614158277}    \\ \cline{2-3} 
                                 & O-17     & \num{1.741113682662514e-05}  \\ \cline{2-3} 
                                 & O-18     & \num{9.189772587857765e-05}  \\ \cline{2-3} 
                                 & U-235    & \num{0.0005581382302893396}  \\ \cline{2-3} 
                                 & U-238    & \num{0.022404154012604437}   \\ \hline
\multirow{5}{*}{MOX-87 Fuel}     & O-16     & \num{0.04585265389377734}    \\ \cline{2-3} 
                                 & O-17     & \num{1.7419604031574338e-05} \\ \cline{2-3} 
                                 & O-18     & \num{9.19424166352541e-05}   \\ \cline{2-3} 
                                 & U-235    & \num{0.0007217486041189947}  \\ \cline{2-3} 
                                 & U-238    & \num{0.02224950230720295}    \\ \hline
\multirow{5}{*}{Guide Tube}      & H-1      & \num{0.050347844752850625}   \\ \cline{2-3} 
                                 & H-2      & \num{7.842394716362082e-06}  \\ \cline{2-3} 
                                 & O-16     & \num{0.025117935412784034}   \\ \cline{2-3} 
                                 & O-17     & \num{9.542402714463945e-06}  \\ \cline{2-3} 
                                 & O-18     & \num{5.03657582849965e-05}   \\ \hline
\multirow{5}{*}{Fission Chamber} & H-1      & \num{0.050347844752850625}   \\ \cline{2-3} 
                                 & H-2      & \num{7.842394716362082e-06}  \\ \cline{2-3} 
                                 & O-16     & \num{0.025117935412784034}   \\ \cline{2-3} 
                                 & O-17     & \num{9.542402714463945e-06}  \\ \cline{2-3} 
                                 & O-18     & \num{5.03657582849965e-05}   \\ \hline
\multirow{12}{*}{Control Rod}    & Ag-107   & \num{0.023523285675833942}   \\ \cline{2-3} 
                                 & Ag-109   & \num{0.02185429814297804}    \\ \cline{2-3} 
                                 & In-113   & \num{0.0003421922042655644}  \\ \cline{2-3} 
                                 & In-115   & \num{0.007651085167039375}   \\ \cline{2-3} 
                                 & Cd-106   & \num{3.38816276451386e-05}   \\ \cline{2-3} 
                                 & Cd-108   & \num{2.4166172970990425e-05} \\ \cline{2-3} 
                                 & Cd-110   & \num{0.0003393605596264083}  \\ \cline{2-3} 
                                 & Cd-111   & \num{0.0003482051612205208}  \\ \cline{2-3} 
                                 & Cd-112   & \num{0.0006561061533306398}  \\ \cline{2-3} 
                                 & Cd-113   & \num{0.00033274751904988726} \\ \cline{2-3} 
                                 & Cd-114   & \num{0.0007825159207295705}  \\ \cline{2-3} 
                                 & Cd-116   & \num{0.00020443276053837845} \\ \hline
\multirow{5}{*}{Moderator}       & H-1      & \num{0.050347844752850625}   \\ \cline{2-3} 
                                 & H-2      & \num{7.842394716362082e-06}  \\ \cline{2-3} 
                                 & O-16     & \num{0.025117935412784034}   \\ \cline{2-3} 
                                 & O-17     & \num{9.542402714463945e-06}  \\ \cline{2-3} 
                                 & O-18     & \num{5.03657582849965e-05}   \\ \hline
\caption{Materials used in C5CE problem}
\label{tab:c5ce}\\
\end{longtable}