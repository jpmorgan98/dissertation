%!TEX root = thesis.tex

\chapter{Introduction}
\label{part:intro}

\epigraphhead[10]{\singlespacing
\epigraph{
    We loiter in the winter \\
    while it is already spring.
	}{Henry David Thoreau}
}

\section{Motivation}

\section{Neutron transport}

Transport equations are, at their most fundamental, equations of continuity.
Transport equations are ubiquitous in nature, describing a number of physical systems including fluid dynamics, chemical reactions, electromagnetism, energy, and heat transfer.
My research contained herein focuses on neutron radiation transport specifically, but could easily be applied to other neutral particles, most relevantly photons \cite{radheattrans2003, chandrasekhar1960radiative}.
In fact many of the methods of solution described in this document started as approximations for radiative heat transfer or photon transport.
%justifying my ME degree
Assuming no neutrons are being produced by fission, the neutron transport equation (NTE) takes the form of an intergro-partial differential Boltzmann-type equation with seven independent variables \cite{duderstadt_hamilton}.
As with any other continuity equation it is written as a set of sources (on the right) and sinks (on the left):
\begin{multline}
    \label{eq:fullNTE}
    \frac{1}{v(E)}\frac{\partial \psi(\boldsymbol{r}, E, \boldsymbol{\hat{\Omega}},t)}{\partial t} + \boldsymbol{\hat{\Omega}} \cdot \nabla \psi(\boldsymbol{r}, E, \boldsymbol{\hat{\Omega}},t) + \Sigma(r, E, t) \psi(\boldsymbol{r}, E, \boldsymbol{\hat{\Omega}},t) = \\
    \int_{4\pi}\int_{0}^{\infty}\Sigma_s(\boldsymbol{r}, E'\rightarrow E, \boldsymbol{\hat{\Omega}'} \rightarrow \boldsymbol{\hat{\Omega}}, t)
    \psi(\boldsymbol{r}, E, \boldsymbol{\hat{\Omega}},t) dE' d\boldsymbol{\hat{\Omega}'} +
    s(\boldsymbol{r}, E, \boldsymbol{\hat{\Omega}},t) \;,
\end{multline}
where $\psi$ is the angular flux, $v$ is the velocity of the particles, $\Sigma$ is the macroscopic total material cross section, $\Sigma_s$ is the macroscopic scattering cross section, $\boldsymbol{r}$ is the location of the particle in three-dimensional space, $\boldsymbol{\hat{\Omega}}$ is the direction of travel in three-dimensional space, $s$ is the isotropic material source of new particles being produced, $t$ is the time, and $E$ is the energy of the particles for $\boldsymbol{r} \in V$, $\boldsymbol{\hat{\Omega}} \in 4\pi$, $0<E<\infty$, and $0<t$ \cite{duderstadt_hamilton}. We also prescribe the initial condition
\begin{equation}
    \psi(\boldsymbol{r}, E, \boldsymbol{\hat{\Omega}},0) = \psi_{initial}(\boldsymbol{r}, E, \boldsymbol{\hat{\Omega}})
\end{equation}
and the boundary condition
\begin{equation}
    \psi(\boldsymbol{r}, E, \boldsymbol{\hat{\Omega}},t) = \psi_{bound}(\boldsymbol{r}, E, \boldsymbol{\hat{\Omega}},t) \text{ for } \boldsymbol{r} \in \partial V \text{ and } \boldsymbol{\hat{\Omega}} \cdot \boldsymbol{n} < 0 \;.
\end{equation}
A number of additional implicit assumptions are needed for the validity of this equation and my research including: particle--particle interactions are rare and can be neglected, neutrons are points in space with no volume, collision events occur instantaneously, and nuclear properties are known \cite{lewis_computational_1984}.
This equation is commonly solved using both deterministic and Monte Carlo solution methods as analytic solutions are sparse.


\section{Document Layout}



\section{Research Questions}