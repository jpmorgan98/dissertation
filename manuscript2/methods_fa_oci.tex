


\newcommand{\exi}{e^{i\lambda\Sigma x_j}}
\newcommand{\omlp}{\omega^{(l+1)}}
\newcommand{\oml}{\omega^{(l)}}
\newcommand{\dx}{\Delta x}
\newcommand{\dt}{\Delta t}
\newcommand{\scatsum}{\sum^{M}_{n=0}}


To study the impact of time dependence on the convergence of an OCI iteration, we conduct a Fourier analysis on the error equation of an infinite-homogeneous medium model problem in slab geometry in a single time step.
Similar to the analysis in the previous section, we can assert that for an iteration scheme
\begin{equation}
    \Psi^{(l+1)} = \bm{T} \Psi^{(l)} \;,
\end{equation}
where $(l)$ is the iteration counter, convergence rate is
\begin{equation}
   \rho = \sup(|\lambda_{\bm{T}}|)\;,
\end{equation}
 where $\lambda_{\bm{T}}$ contains the eigenvalues of $\bm{T}$ \cite{golub_matrix_1983, isaacson_numerical_1966}.
An iterative method will converge if and only if $\rho<1$. 
Furthermore, iterations converge faster for smaller $\rho$.

To derive the transport matrix $\bm{T}$ we can again use Fourier separation analysis on a model problem.
We first start by describing the absolute error of the angular flux at iteration step $(l)$
\begin{equation}
    \mathbf{f}^l = \Psi^{\text{converged}} - \Psi^l \;,
\end{equation}
and our Fourier anzats on a functional form of that error
\begin{subequations}
    \label{eq:anz}
    \begin{align}
        f^{(l)}_{m,k,j,L/R} &= \omega^{(l)}a_{m,L/R}e^{i\lambda\Sigma x_j} \; ,
        &
        f^{(l)}_{m,k+1/2,j,L/R} &= \omega^{(l)}b_{m,L/R}e^{i\lambda\Sigma x_j} \;.
    \end{align}
\end{subequations} 
The upstream closures at the left boundary of the cell are
\begin{subequations}
\begin{equation}
    f_{m,k,j-1/2} =
    \begin{cases}
        f_{m,k,j-1, R} \;, & \mu > 0 \\
        f_{m,k,j, L} \;, & \mu < 0 
    \end{cases} \;,
\end{equation}
and at the right are
\begin{equation}
    f_{m,k,j+1/2} =
    \begin{cases}
        f_{m,k,j, R} \;, & \mu > 0 \\
        f_{m,k,j+1, L} \;, & \mu < 0 
    \end{cases} \;.
\end{equation}
\end{subequations}

\iffalse
\begin{subequations}
    \begin{align}
        f^{(l)}_{m,k,j-1/2} &= f^{(l)}_{m,j-1,R} \; ,
        &
        f^{(l+1)}_{m,k,j+1/2} &= f^{(l+1)}_{m,k,j,R} \; ,
    \end{align}
    \begin{align}
        f^{(l)}_{m,k+1/2,j-1/2} &= f^{(l)}_{m,k+1/2,j-1,R} \; ,
        &
        f^{(l+1)}_{m,k+1/2,j+1/2} &= f^{(l+1)}_{m,k+1/2,j,R} \; ,
    \end{align}
\end{subequations}
where $i=\sqrt{-1}$. Note only the \textit{incident} flux is lagged.
Similarly, for negative directions ($\mu_m<0$):
\begin{subequations}
    \begin{align}
        f^{(l+1)}_{m,k,j-1/2} &= f^{(l+1)}_{m,j,L} \; ,
    &
        f^{(l)}_{m,k,j+1/2} &= f^{(l)}_{m,k,j+1,L} \; ,
    \end{align}
    \begin{align}
        f^{(l+1)}_{m,k+1/2,j-1/2} &= f^{(l+1)}_{m,k+1/2,j,L} \; ,
        &
        f^{(l)}_{m,k+1/2,j+1/2} &= f^{(l)}_{m,k+1/2,j+1,R} \; .
    \end{align}
\end{subequations}
\fi

Now, substitute the ansätz and upstream closures into the error form of Eq. \eqref{eq:scb-mb} and derive the eigensystem.
This is done by (1) collecting like terms, (2) dividing both sides by $\omega^{(l)} e^{i\Sigma x_j}$, (3) isolating terms with a remaining $\omega$ to the left-hand side, and finally (4) forming the eigensystem into the iteration matrix over all angular directions:
\begin{equation}
    \bm{T}_{OCI} = \left( 
    \bm{L}_c
    - \bm{S}
    \right)^{-1}
    \begin{bmatrix}
        \bm{L}_b^- & 0\\
        0 & \bm{L}_b^+
    \end{bmatrix} \;,
\end{equation}
which now forms a well-posed eigenvalue problem over all angles:
\begin{equation}
    \lambda\bm{a} = \bm{T} \bm{a} \; ,
\end{equation}
where the eigenvector $\mathbf{a}$ is defined by
\begin{equation}
    \mathbf{a} = \begin{bmatrix}
        a_{1} & a_{2} & \cdots & a_M
    \end{bmatrix} ^T \;,
\end{equation}
\begin{equation}
    a_m = \begin{bmatrix}
        a_{mR} & a_{mL} & b_{mR} & b_{mL} 
    \end{bmatrix} ^T \; ,
\end{equation}
$\bm{L}_c$ is the linear within-cell transport operator defined by Eqs.~\eqref{eq:Aja} and \eqref{eq:Aj}, 
\begin{equation}
    \bm{L}_b^+ = 
    \begin{bmatrix}
        0 & 0 & 0 & 0 \\
        -\mu_m e^{-i\lambda\sigma\dx} & 0 & 0 & 0 \\
        0 & 0 & 0 & 0 \\
        0 & 0 & -\mu_m e^{-i\lambda\sigma\dx} & 0
    \end{bmatrix} \; ,
\end{equation}
and
\begin{equation}
    \bm{L}_b^- = 
    \begin{bmatrix}
        0 & \mu_m e^{i\lambda\sigma\dx} & 0 & 0 \\
        0 & 0 & 0 & 0 \\
        0 & 0 & 0 & \mu_m e^{i\lambda\sigma\dx} \\
        0 & 0 & 0 & 0
    \end{bmatrix} \; .
\end{equation}
The scattering matrix is again akin to the previously described transport matrix in Eq.~\eqref{eq:scatter}.
Finally, to numerically evaluate the spectral radius we form the system for a given set of angles and weights from Gauss--Legendre quadrature and solve with \texttt{numpy.max(numpy.abs(numpy.linalg.eig(T)))} for $\omega \in [0,2\pi]$ at discrete points.
We vary the cellular optical thickness ($\delta =\Sigma \Delta x$), mean free time ($\tau = \Sigma v\Delta t$), and scattering ratio ($c=\Sigma_s/\Sigma$) to study convergence behavior in various physical regimes. 
The analogous eigensystem for source iteration is
\begin{equation}
    \bm{T}_{SI} = \left( 
    \bm{L}_c
    + \begin{bmatrix}
        \bm{L}_b^- & 0\\
        0 & \bm{L}_b^+
    \end{bmatrix}
    \right)^{-1}
    \bm{S} \; .
\end{equation} 
Section~\ref{sec:results-faoci} contains the results of this analysis.
